\let\negmedspace\undefined
\let\negthickspace\undefined
\documentclass[journal]{IEEEtran}
\usepackage[a5paper, margin=10mm, onecolumn]{geometry}
\usepackage{lmodern} % Ensure lmodern is loaded for pdflatex
\usepackage{tfrupee} % Include tfrupee package

\setlength{\headheight}{1cm} % Set the height of the header box
\setlength{\headsep}{0mm}     % Set the distance between the header box and the top of the text

\usepackage{gvv-book}
\usepackage{gvv}
\usepackage{cite}
\usepackage{amsmath,amssymb,amsfonts,amsthm}
\usepackage{algorithmic}
\usepackage{graphicx}
\usepackage{textcomp}
\usepackage{xcolor}
\usepackage{txfonts}
\usepackage{listings}
\usepackage{enumitem}
\usepackage{mathtools}
\usepackage{gensymb}
\usepackage{comment}
\usepackage[breaklinks=true]{hyperref}
\usepackage{tkz-euclide} 
\usepackage{listings}
\usepackage{gvv}                                       
\def\inputGnumericTable{}                                 
\usepackage[latin1]{inputenc}                                
\usepackage{color}                                            
\usepackage{array}                                            
\usepackage{longtable}                                       
\usepackage{calc}                                             
\usepackage{multirow}                                         
\usepackage{hhline}                                           
\usepackage{ifthen}                                           
\usepackage{lscape}
\begin{document}

\bibliographystyle{IEEEtran}
\vspace{3cm}

\title{2022-July Session-07-26-2022-shift-1-1-15}
\author{AI24BTECH11019-KOTHA PRATHEEK REDDY}

 \maketitle
%\newpage
% \bigskip
{\let\newpage\relax\maketitle}

\renewcommand{\thefigure}{\theenumi}
\renewcommand{\thetable}{\theenumi}
\setlength{\intextsep}{10pt} % Space between text and floats


\numberwithin{equation}{enumi}
\numberwithin{figure}{enumi}
\renewcommand{\thetable}{\theenumi}
\begin{enumerate}
	\item Let $f: \mathbb{R} \rightarrow \mathbb{R}$ be a continuous function such that $f\brak{3x} - f\brak{x} = x$. If $f\brak{8} = 7$, then $f\brak{14}$ is equal to \hfill{[July 2021]}
    \begin{multicols}{2}
    \begin{enumerate}
        \item 4
        \item 10
        \item 11
        \item 16
    \end{enumerate}
     \end{multicols} 
     \item Let $O$ be the origin and $A$ be the point $z_1 = 1 + 2 i$. If $B$ is the point $z_2$, $Re\brak{z_2} < 0$,such that $OAB$ is a right angled isosceles triangle with $OB$ as hypotenuse, then which of the following is NOT true? \hfill{[July 2021]}
     \begin{multicols}{2}
     \begin{enumerate}
         \item arg $z_2$ = $\pi - \tan^{-1}{3}$
         \item arg $\brak{z_1 - 2 z_2} = -\tan^{-1}\frac{4}{3}$
         \item $|z_2 | = \sqrt{10}$
         \item $|2z_1 - z_2| = 5$
     \end{enumerate}
         
     \end{multicols}
     \item If the system of linear equations.
     \begin{align}
         8x+y+4z &= -2         \notag\\ 
         x+y+z &= 0             \notag\\
         \lambda x-3y &= \mu     \notag
     \end{align}
     has infinitely many solutions, then the distance of the point $\brak{\lambda,\mu,-\frac{1}{2}}$ from the plane $8x +y+4z+2=0$ is \hfill{[July 2021]}
     \begin{multicols}{2}
     \begin{enumerate}
         \item $3\sqrt{5}$
         \item $4$
         \item $\frac{26}{9}$
         \item $\frac{10}{3}$
     \end{enumerate}
        \end{multicols}
    \item Let $A$ be a $2 \times 2$ matrix with $\det\brak{A}= -1$ and $\det\brak{\brak{A+I}\brak{Adj\brak{A}+I}} = 4$. Then the sum of the diagonal elements of $A$ can be: \hfill{[July 2021]}
    \begin{multicols}{2}
        \begin{enumerate}
            \item $-1$
            \item $2$
            \item $1$
            \item $-\sqrt{2}$
            
        \end{enumerate}
    \end{multicols}
    \item The odd natural number $a$, such that the area of the region bounded by $y = 1,y=3,x=0,x=y^a$ is $\frac{364}{3}$, equal to: \hfill{[July 2021]}
    \begin{multicols}{2}
        \begin{enumerate}
            \item $3$
            \item $5$
            \item$7$
            \item $9$
        \end{enumerate}
    \end{multicols}
    \item Consider two G.Ps. $2,2^2,2^3,\dots$ and $4,4^2,4^3,\dots$ of $60$ and $n$ terms respectively. If the geometric mean of $60+n$ terms is $\brak{2}^\frac{225}{8}$, then $\sum_{k=1}^{n} k\brak{n-k}$ is equal to: \hfill{[July 2021]}
    \begin{multicols}{2}
        \begin{enumerate}
            \item 560
            \item 1540
            \item 1330
            \item 2600
        \end{enumerate}
    \end{multicols}
    \item If the function 
    $f\brak{x} = 
    \begin{cases}
        \frac{\log_e\brak{1-x+x^2}+\log_e\brak{1+x+x^2}}{\sec{x}-\cos{x}}, &x \in \brak{\frac{-\pi}{2},\frac{\pi}{2}} \\
    k        ,&x = 0
    \end{cases}$\\
    is continuous at $x=0$, then $k$ is equal to: \hfill{[July 2021]}
    \begin{multicols}{2}
        \begin{enumerate}
            \item $1$
            \item $-1$
            \item$e$
            \item $0$
            
        \end{enumerate}
    \end{multicols}
    \item If $ f\brak{x} =
        \begin{cases}
            x+a , &x \leq 0 \\
            |x-4|, &x > 0
        \end{cases} $ 
        and
        
        $ g\brak{x} = 
        \begin{cases}
            x+1 , &x<0 \\
            \brak{x-4}^2 + b,&x\geq 0
        \end{cases}
        $
    
     are continuous and $\mathbb{R}$, then $\brak{f \circ g}\brak{2} + \brak{f\circ g}\brak{-2}$ is equal to: \hfill{[July 2021]}
    \begin{multicols}{2}
        \begin{enumerate}
            \item $-10$
            \item $10$
            \item $8$
            \item $-8$
        \end{enumerate}
    \end{multicols}
    \item Let $
        f\brak{x} = 
            \begin{cases}
                x^3 - x^2 + 10x - 7, &x\leq 1 \\
                -2x + \log_{2} \brak{b^2 - 4},&x>1
            \end{cases}
        $
        Then the set of all values of $b$, for which $f\brak{x}$ has maximum value at $x=1$, is: \hfill{[July 2021]}
        \begin{multicols}{2}
            \begin{enumerate}
                \item $\brak{-6,-2}$
                \item \brak{2,6}
                \item $[-6,-2)\cup(2,6]$
                \item $[-\sqrt{6},-2)\cup(2,\sqrt{6}]$
            \end{enumerate}
        \end{multicols}
        \item If $a = \lim_{n \to \infty} \sum_{k=1}^{n} \frac{2n}{n^2 + k^2}$ and $f\brak{x} = \sqrt{\frac{1 - \cos{x}}{1 + \cos{x}}},x \in \brak{0,1}$, then: \hfill{[July 2021]}
        \begin{multicols}{2}
            \begin{enumerate}
                \item $2\sqrt{2}f\brak{\frac{a}{2}} = f^{\prime}\brak{\frac{a}{2}}$
                \item $f\brak{\frac{a}{2}} f^{\prime}\brak{\frac{a}{2}} = \sqrt{2}$
                \item $\sqrt{2}f\brak{\frac{a}{2}} = f^{\prime}\brak{\frac{a}{2}}$
                \item $f\brak{\frac{a}{2}} = \sqrt{2}f^{\prime}\brak{\frac{a}{2}}$
            \end{enumerate}
        \end{multicols}
        \item $\frac{dy}{dx} + 2y \tan x = \sin x, 0<x<\frac{\pi}{2}$ and $y\brak{\frac{\pi}{3}} = 0$, then the maximum value of $y\brak{x}$ is \hfill{[July 2021]}
        \begin{multicols}{2}
            \begin{enumerate}
                \item $\frac{1}{8}$
                \item $\frac{3}{4}$
                \item $\frac{1}{4}$
                \item $\frac{3}{8}$
            \end{enumerate}
        \end{multicols}

        \item A point $P$ moves so that the sum of squares of its distances from the points $\brak{1,2}$ and $\brak{-2,1}$ is $14$. Let $f\brak{x,y} = 0$ be the locus of $P$, which intersects the x-axis at the points $A$, $B$ and the y-axis at the points $C$, $D$. Then the area of the quadrilateral $ABCD$ is equal to \hfill{[July 2021]}
        \begin{multicols}{2}
            \begin{enumerate}
                \item $\frac{9}{2}$
                \item $\frac{3\sqrt{17}}{2}$
                \item $\frac{3\sqrt{17}}{4}$
                \item $9$
                
            \end{enumerate}
        \end{multicols}
        \item Let the tangent drawn to the parabola $y^2 = 24x$ at the point $\brak{\alpha,\beta}$ is perpendicular to the line $2x+2y = 5$. Then the normal to the hyperbola $\frac{x^2}{\alpha^2} - \frac{y^2}{\beta^2} = 1$ at the point $\brak{\alpha+4,\beta+4}$ does NOT pass through the point: \hfill{[July 2021]}
        \begin{multicols}{2}
            \begin{enumerate}
                \item $\brak{25,10}$
                \item $\brak{20,12}$
                \item $\brak{30,8}$
                \item $\brak{15,13}$
            \end{enumerate}
        \end{multicols}
        \item The length of the perpendicular from the point $\brak{1,-2,5}$ on the line passing through $\brak{1,2,4}$ and parallel to the line $x+y-z = 0 = x-2y+3z-5$ is: \hfill{[July 2021]}
        \begin{multicols}{2}
            \begin{enumerate}
                \item $\sqrt{\frac{21}{2}}$
                \item $\sqrt{\frac{9}{2}}$
                \item $\sqrt{\frac{73}{2}}$
                \item $1$
            \end{enumerate}
        \end{multicols}
       \item  Let $\vec{a}=\alpha \hat{i} + \hat{j} - \hat{k}$ and $\vec{b} = 2\hat{i}+\hat{j}-\alpha\hat{k}$. If the projection of $\vec{a} \times \vec{b}$ on the vector $-\hat{i}+2\hat{j}-2\hat{k}$ is 30, then $\alpha$ is equal to \hfill{[July 2021]}
       \begin{multicols}{2}
           \begin{enumerate}
               \item $\frac{15}{2}$
               \item 8
               \item $\frac{13}{2}$
               \item $7$
           \end{enumerate}
       \end{multicols}
    
\end{enumerate}
\end{document}
