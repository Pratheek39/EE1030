\let\negmedspace\undefined
\let\negthickspace\undefined
\documentclass[journal]{IEEEtran}
\usepackage[a5paper, margin=10mm, onecolumn]{geometry}
\usepackage{lmodern} % Ensure lmodern is loaded for pdflatex
\usepackage{tfrupee} % Include tfrupee package

\setlength{\headheight}{1cm} % Set the height of the header box
\setlength{\headsep}{0mm}     % Set the distance between the header box and the top of the text

\usepackage{gvv-book}
\usepackage{gvv}
\usepackage{cite}
\usepackage{amsmath,amssymb,amsfonts,amsthm}
\usepackage{algorithmic}
\usepackage{graphicx}
\usepackage{textcomp}
\usepackage{xcolor}
\usepackage{txfonts}
\usepackage{listings}
\usepackage{enumitem}
\usepackage{mathtools}
\usepackage{gensymb}
\usepackage{comment}
\usepackage[breaklinks=true]{hyperref}
\usepackage{tkz-euclide} 
\usepackage{listings}
\usepackage{gvv}                                       
\def\inputGnumericTable{}                                 
\usepackage[latin1]{inputenc}                                
\usepackage{color}                                            
\usepackage{array}                                            
\usepackage{longtable}                                       
\usepackage{calc}                                             
\usepackage{multirow}                                         
\usepackage{hhline}                                           
\usepackage{ifthen}                                           
\usepackage{lscape}
\begin{document}

\bibliographystyle{IEEEtran}
\vspace{3cm}

\title{1.11.8}
\author{AI24BTECH11019-KOTHA PRATHEEK REDDY}

 \maketitle
%\newpage
% \bigskip
{\let\newpage\relax\maketitle}

\renewcommand{\thefigure}{\theenumi}
\renewcommand{\thetable}{\theenumi}
\setlength{\intextsep}{10pt} % Space between text and floats


\numberwithin{equation}{enumi}
\numberwithin{figure}{enumi}
\renewcommand{\thetable}{\theenumi}
\textbf{Question}:\\
Find the direction cosines of the line joining points $\vec{P}$  \myvec{4,3,-5} and $\vec{Q}$ \myvec{ -2, 1, 8}.\\

\solution
Let the unit vector in the direction of the vector $\vec{PQ}$ be $\hat{a}$.Then \\
\begin{align}
    \hat{a} &= \frac{\vec{Q}-\vec{P}}{||\vec{Q}-\vec{P}||} \\
    \vec{P} &= \myvec{4 \\ 3 \\ -5} \\
    \vec{Q} &= \myvec{-2 \\ 1 \\8} \\
    \vec{Q}-\vec{P} &= \myvec{-6 \\ -2 \\ 13} \\
	||\vec{Q}-\vec{P}|| &= \sqrt{\brak{-6}^2 + \brak{-2}^2 + 13^2} \notag \\
    &= \sqrt{209} 
\end{align}
From the above equations, \\
\begin{align}
    \hat{a} = \myvec{ \frac{-6}{\sqrt{209}} \\ \frac{-2}{\sqrt{209}} \\ \frac{13}{\sqrt{209}}}
\end{align}
The direction vectors of the the line joining $\vec{A}$ and $\vec{B}$ are the elements of $\hat{a}$ i.e. $\frac{-6}{\sqrt{209}}$ , $\frac{-2}{\sqrt{209}} , \frac{13}{\sqrt{209}}$
 \begin{table}[H]
      \centering
      \begin{tabular}{|c|c|}
   \hline
   Point & Coordinate \\
   \hline
	$\vec{P}$ & $\myvec{4,3,-5}$ \\
   \hline
	$\vec{Q}$ & $\myvec{-2,1,8}$ \\
   \hline
\end{tabular}

      \caption{Coordinates}
      \label{}
  \end{table}
  \begin{figure}[h!]
   \centering
   \includegraphics[width=\linewidth]{Figures/Figure_1.png}
	  \caption{Line joining $\vec{P}$ and $\vec{Q}$}
   \label{stemplot}
\end{figure}




\end{document}
