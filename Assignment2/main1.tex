%iffalse
\let\negmedspace\undefined
\let\negthickspace\undefined
\documentclass[journal,12pt,onecolumn]{IEEEtran}
\usepackage{cite}
\usepackage{amsmath,amssymb,amsfonts,amsthm}
\usepackage{algorithmic}
\usepackage{graphicx}
\usepackage{textcomp}
\usepackage{xcolor}
\usepackage{txfonts}
\usepackage{listings}
\usepackage{enumitem}
\usepackage{mathtools}
\usepackage{gensymb}
\usepackage{comment}
\usepackage[breaklinks=true]{hyperref}
\usepackage{tkz-euclide} 
\usepackage{listings}
\usepackage{gvv}                                        
%\def\inputGnumericTable{}                                 
\usepackage[latin1]{inputenc}                                
\usepackage{color}                                            
\usepackage{array}                                            
\usepackage{longtable}                                       
\usepackage{calc}                                             
\usepackage{multirow}                                         
\usepackage{hhline}                                           
\usepackage{ifthen}                                           
\usepackage{lscape}
\usepackage{tabularx}
\usepackage{array}
\usepackage{float}
\usepackage{multicol}
\usepackage{dashrule}

\newtheorem{theorem}{Theorem}[section]
\newtheorem{problem}{Problem}
\newtheorem{proposition}{Proposition}[section]
\newtheorem{lemma}{Lemma}[section]
\newtheorem{corollary}[theorem]{Corollary}
\newtheorem{example}{Example}[section]
\newtheorem{definition}[problem]{Definition}
\newcommand{\BEQA}{\begin{eqnarray}}
\newcommand{\EEQA}{\end{eqnarray}}
\newcommand{\define}{\stackrel{\triangle}{=}}
\theoremstyle{remark}
\newtheorem{rem}{Remark}

% Marks the beginning of the document
\begin{document}
\bibliographystyle{IEEEtran}
\vspace{3cm}

\title{Assignment-2}
\author{AI24BTECH11019-PRATHEEK}
\maketitle

\bigskip

\renewcommand{\thefigure}{\theenumi}
\renewcommand{\thetable}{\theenumi}

\section*{A.Fill in the blanks}
\begin{enumerate}
    \item If $y$=$f{\brak{\frac{2x+1}{x^2+1}}}$ and $f^{\prime}\brak{x}$ =$\sin{x}^2$, then $\frac{dy}{dx}$ = \dots\hfill \brak{1982-2 Marks} 
    \item If $f_r\brak{x}$,$g_r\brak{x}$ ,$h_r\brak{x}$ , $r=1,2,3$ are polynomials in x such that  $f_r\brak{a}$=$g_r\brak{a}$=$h_r\brak{a}$,$r=1,2,3$ and \\$F(x)$ = 


    $\mydet{
			f_1\brak{x} & f_2\brak{x} & f_3\brak{x} \\
			g_1\brak{x} & g_2\brak{x} & g_3\brak{x}\\
			h_1\brak{x} & h_2\brak{x}& h_3\brak{x}
		}  $
    then $F^{\prime}\brak{x}$ at $x=a$ is \dots $\hfill\brak{1985 -2Marks}$
\item If $f(x)=\log_{x}\brak{\ln{x}}$,then $f^{\prime}\brak{x}$ at $x=e$ is\dots \hfill \brak{1982-2 Marks} 
\item The derivative of $\sec^{-1}{\brak{\frac{1}{2x^2-1}}}$ with respect to $\sqrt{1-x^2}$
at $x=\frac{1}{e}$ is \dots $\hfill\brak{1986 -2Marks}$
\item If $f(x)$ = $|x-2|$ and $g(x)=f[f(x)]$,then $g^{\prime}\brak{x}$= \dots for $x>20$ $\hfill\brak{1990-2 Marks}$  
\item if $xe^{xy}=y+\sin^2{x}$,then at $x=0$, $\frac{dy}{dx}$ = \dots\\.$\hfill \brak{1992-1Mark}$
\end{enumerate}
\section*{B.TRUE/FALSE}
\begin{enumerate}
    \item The derivative of an even function is always an odd function $\hfill\brak{1983-1 Mark}$
\end{enumerate}
\section*{C.MCQs with One Correct Answer}
\begin{enumerate}
    \item If $y=P\brak{x}$, a polynomial of degree 3,then $2\frac{d}{dx}\brak{y^3\frac{d^2y}{dx^2}}$ equals
    $\hfill\brak{1988-2 Marks}$
\begin{enumerate}
\begin{multicols}{2}
    \item $P^{\prime\prime}\brak{x}+P^{\prime}\brak{x}$
    \item $P^{\prime}\brak{x}P^{\prime\prime}\brak{x}$
    \item $P\brak{x}P^{\prime\prime}\brak{x}$
    \item a constant
 \end{multicols}   
\end{enumerate}
\item Let $f\brak{x}$ be a qaudratic expression which is positive for all the real values of $x$. If $g\brak{x} = f\brak{x}+f^{\prime}\brak{x}+f^{\prime\prime}\brak{x}$,then for any real $x$,
\begin{enumerate}
\begin{multicols}{2}
    \item $g\brak{x}<0$
    \item $g\brak{x}>0$
    \item $g\brak{x}=0$
    \item $g\brak{x}\geq0$
\end{multicols}
\end{enumerate}
\item If $y=\brak{\sin{x}}^{\tan{x}}$ then $\frac{dy}{dx}$ is equal to \\.\hfill\brak{1994}
\begin{enumerate}
    \item $\brak{\sin{x}}^{\tan{x}}\brak{1+\sec^2{\log{\sin{x}}}}$
    \item $\tan{x}\brak{\sin{x}}^{\tan{x}-1}.\cos{x}$
    \item $\brak{\sin{x}}^{\tan{x}}\sec^2{\log{\sin{x}}}$
    \item $\tan{x}\brak{\sin{x}}^{\tan{x}-1}$
\end{enumerate}
\item If $x^2+y^2=1$ then \hfill\brak{2000}
\begin{enumerate}
\begin{multicols}{2}
    \item $yy^{\prime\prime}-2\brak{y^{\prime}}^2+1=0$
    \item $yy^{\prime\prime}+\brak{y^{\prime}}^2+1=0$
    \item $yy^{\prime\prime}-\brak{y^{\prime}}^2+1=0$
    \item $yy^{\prime\prime}+2\brak{y^{\prime}}^2+1=0$
    \end{multicols}
\end{enumerate}
\item Let $f\brak{x}:\brak{0,\infty}\rightarrow \mathbb{R}$ and $F\brak{x}=\int_{0}^x f\brak{t}dt$.If $F(x^2)=x^2\brak{1+x}$, then $f(4)$ equals $\hfill\brak{2001S}$
\begin{enumerate}
\begin{multicols}{4}
    \item $\frac{5}{4}$
    \item 7
    \item 4
    \item 2
    \end{multicols}
\end{enumerate}
\item If $y$ is a function of $x$ and $\log{\brak{x+y}}-2xy=0$,then the value of $y^{\prime}(0)$ is equal to $\hfill\brak{2004S}$
\begin{enumerate}
\begin{multicols}{4}
    \item 1
    \item $-1$
    \item 2
    \item 0
\end{multicols}

\end{enumerate}
\end{enumerate}

\end{document}

