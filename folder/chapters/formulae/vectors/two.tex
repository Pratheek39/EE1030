%\begin{enumerate}[label=\arabic*.,ref=\theenumi]
\begin{enumerate}[label=\thesubsection.\arabic*.,ref=\thesubsection.\theenumi]
\numberwithin{equation}{enumi}
	\item The {\em direction vector} of $AB$ is defined as
		\begin{align}
		\label{eq:dir-vec}
			\vec{m}=\vec{B}-
			\vec{A} = \kappa
			\myvec{1 \\ m}
		\end{align}
		where $m$ is the slope of $AB$.  We also say that 
\begin{align}
\vec{m}\equiv   \myvec{1 \\ m}
\end{align}
	\item The lines with direction vectors $\vec{m}_1$ and $\vec{m}_2$
		respectively, are parallel if 
\begin{align}
\vec{m}_1\equiv   \vec{m}_2
\end{align}
  \item If $ABCD$ be a parallelogram with $AB \parallel CD$,
	  \label{prop:two-pgm}
  \begin{align}
	  \label{eq:two-pgm}
 \vec{B}-\vec{A} = \vec{C} -\vec{D}
  \end{align}
\item If $\vec{D}$ divides $BC$ in the ratio $k : 1$,
		\begin{align}
			\vec{D}= \frac{k\vec{C}+\vec{B}}{k+1}
	  \label{eq:section_formula}
		\end{align}
  \item 
If $PQRS$ is formed by joining the mid points of $ABCD$, 
\begin{align}
  \vec{P} = \frac{1}{2}\brak{\vec{A}+\vec{B}} 
  ,\,
 \vec{Q} = \frac{1}{2}\brak{\vec{B}+\vec{C}} 
 \\
 \vec{R} = \frac{1}{2}\brak{\vec{C}+\vec{D}}   
  ,\,
 \vec{S} = \frac{1}{2}\brak{\vec{D}+\vec{A}}  
 \\
	\implies 
 \vec{P}-\vec{Q} = \vec{S} -\vec{R}.
  \label{eq:10/7/4/8det2f}
\end{align}
Hence, $PQRS$ is a parallelogram
	  from \eqref{eq:two-pgm}.
	\item In 2D space,  the basis vectors are defined as 
\begin{align}
	\vec{e}_1 = \myvec{1 \\ 0},\
	\vec{e}_2 = \myvec{0 \\ 1}.
\end{align}
	\item The length of a vector  is  defined as
		\begin{align}
		\label{eq:side-length}
			 \norm{\vec{x}} \triangleq \sqrt{\vec{x}^{\top}\vec{x}}
		\end{align}
		For example, if 
\begin{align}
\vec{x}
	&=\myvec{3 \\ 4},
	\\
	\vec{x}^{\top}\vec{x} &= \myvec{3 & 4}\myvec{3 \\ 4}
	\label{eq:scalar-product}
	\\
	&=3 \times 3 + 4 \times 4 = 25
\end{align}
yielding
		\begin{align}
			 \norm{\vec{x}} = 5.
		\end{align}
	\eqref{eq:scalar-product}
	is known as the scalar product.
	\item The unit vector in the direction of $\vec{x}$ is 
		\begin{align}
		\label{eq:unit-vec}
			 \frac{\vec{x}}{\norm{\vec{x}}} 
		\end{align}
		\iffalse
\item   For a 2D space, 
	points $\vec{A}, \vec{B}, \vec{C}$ are defined to be collinear if 
		\fi
	\item 
	Points $\vec{A}, \vec{B}, \vec{C}$ are defined to be collinear if 
		\begin{align}
			\label{eq:line-rank-2}
			\rank{\myvec{\vec{B}-\vec{A}& \vec{C}-\vec{A}}} = 1
		\end{align}
	\item 
\begin{align}
			\label{eq:mat-rank-t}
	\rank{\vec{A}}
	=
	\rank{\vec{A}^\top}
\end{align}
\item In the 2D space, the unit direction vector is defined as
\begin{align}
		\label{eq:dir-vec-3d}
\vec{m}=\myvec{\cos \alpha\\ \cos \beta }
\end{align}
where ${ \alpha,  \beta }$ are the angles made by the vector with the axes.
\item Code for plotting points and vector arithmetic
	\begin{lstlisting}
	codes/book/points.py
\end{lstlisting}
\item Code for section formula 
	\begin{lstlisting}
	codes/book/section.py
\end{lstlisting}
\item Code for matrix rank
	\begin{lstlisting}
	codes/book/rank.py
\end{lstlisting}
\item Code for vector length
	\begin{lstlisting}
	codes/book/dist.py
\end{lstlisting}
\end{enumerate}
