\begin{enumerate}[label=\thesection.\arabic*,ref=\thesection.\theenumi]
\item  Show that the three lines with direction cosines
$\frac{12}{13},\frac{-3}{13},\frac{-4}{13}$; $\frac{4}{13},\frac{12}{13},\frac{3}{13}$; $\frac{3}{13},\frac{-4}{13},\frac{12}{13}$; are mutually perpendicular.\\
    \solution
		\iffalse
\documentclass[journal,12pt,twocolumn]{IEEEtran}
%
\usepackage{setspace}
\usepackage{gensymb}
%\doublespacing
\singlespacing

%\usepackage{graphicx}
%\usepackage{amssymb}
%\usepackage{relsize}
\usepackage[cmex10]{amsmath}
%\usepackage{amsthm}
%\interdisplaylinepenalty=2500
%\savesymbol{iint}
%\usepackage{txfonts}
%\restoresymbol{TXF}{iint}
%\usepackage{wasysym}
\usepackage{amsthm}
%\usepackage{iithtlc}
\usepackage{mathrsfs}
\usepackage{txfonts}
\usepackage{stfloats}
\usepackage{bm}
\usepackage{cite}
\usepackage{cases}
\usepackage{subfig}
%\usepackage{xtab}
\usepackage{longtable}
\usepackage{multirow}
%\usepackage{algorithm}
%\usepackage{algpseudocode}
\usepackage{enumitem}
\usepackage{mathtools}
\usepackage{steinmetz}
\usepackage{tikz}
\usepackage{circuitikz}
\usepackage{verbatim}
\usepackage{tfrupee}
\usepackage[breaklinks=true]{hyperref}
%\usepackage{stmaryrd}
\usepackage{tkz-euclide} % loads  TikZ and tkz-base
%\usetkzobj{all}
\usetikzlibrary{calc,math}
\usepackage{listings}
    \usepackage{color}                                            %%
    \usepackage{array}                                            %%
    \usepackage{longtable}                                        %%
    \usepackage{calc}                                             %%
    \usepackage{multirow}                                         %%
    \usepackage{hhline}                                           %%
    \usepackage{ifthen}                                           %%
  %optionally (for landscape tables embedded in another document): %%
    \usepackage{lscape}     
\usepackage{multicol}
\usepackage{chngcntr}
%\usepackage{enumerate}

%\usepackage{wasysym}
%\newcounter{MYtempeqncnt}
\DeclareMathOperator*{\Res}{Res}
%\renewcommand{\baselinestretch}{2}
\renewcommand\thesection{\arabic{section}}
\renewcommand\thesubsection{\thesection.\arabic{subsection}}
\renewcommand\thesubsubsection{\thesubsection.\arabic{subsubsection}}

\renewcommand\thesectiondis{\arabic{section}}
\renewcommand\thesubsectiondis{\thesectiondis.\arabic{subsection}}
\renewcommand\thesubsubsectiondis{\thesubsectiondis.\arabic{subsubsection}}

% correct bad hyphenation here
\hyphenation{op-tical net-works semi-conduc-tor}
\def\inputGnumericTable{}                                 %%

\lstset{
%language=C,
frame=single, 
breaklines=true,
columns=fullflexible
}
%\lstset{
%language=tex,
%frame=single, 
%breaklines=true
%}


\begin{document}
%


\newtheorem{theorem}{Theorem}[section]
\newtheorem{problem}{Problem}
\newtheorem{proposition}{Proposition}[section]
\newtheorem{lemma}{Lemma}[section]
\newtheorem{corollary}[theorem]{Corollary}
\newtheorem{example}{Example}[section]
\newtheorem{definition}[problem]{Definition}
%\newtheorem{thm}{Theorem}[section] 
%\newtheorem{defn}[thm]{Definition}
%\newtheorem{algorithm}{Algorithm}[section]
%\newtheorem{cor}{Corollary}
\newcommand{\BEQA}{\begin{eqnarray}}
\newcommand{\EEQA}{\end{eqnarray}}
\newcommand{\define}{\stackrel{\triangle}{=}}

\bibliographystyle{IEEEtran}
%\bibliographystyle{ieeetr}


\providecommand{\mbf}{\mathbf}
\providecommand{\pr}[1]{\ensuremath{\Pr\left(#1\right)}}
\providecommand{\qfunc}[1]{\ensuremath{Q\left(#1\right)}}
\providecommand{\sbrak}[1]{\ensuremath{{}\left[#1\right]}}
\providecommand{\lsbrak}[1]{\ensuremath{{}\left[#1\right.}}
\providecommand{\rsbrak}[1]{\ensuremath{{}\left.#1\right]}}
\providecommand{\brak}[1]{\ensuremath{\left(#1\right)}}
\providecommand{\lbrak}[1]{\ensuremath{\left(#1\right.}}
\providecommand{\rbrak}[1]{\ensuremath{\left.#1\right)}}
\providecommand{\cbrak}[1]{\ensuremath{\left\{#1\right\}}}
\providecommand{\lcbrak}[1]{\ensuremath{\left\{#1\right.}}
\providecommand{\rcbrak}[1]{\ensuremath{\left.#1\right\}}}
\theoremstyle{remark}
\newtheorem{rem}{Remark}
\newcommand{\sgn}{\mathop{\mathrm{sgn}}}
\providecommand{\abs}[1]{\left\vert#1\right\vert}
\providecommand{\res}[1]{\Res\displaylimits_{#1}} 
\providecommand{\norm}[1]{\left\lVert#1\right\rVert}
%\providecommand{\norm}[1]{\lVert#1\rVert}
\providecommand{\mtx}[1]{\mathbf{#1}}
\providecommand{\mean}[1]{E\left[ #1 \right]}
\providecommand{\fourier}{\overset{\mathcal{F}}{ \rightleftharpoons}}
%\providecommand{\hilbert}{\overset{\mathcal{H}}{ \rightleftharpoons}}
\providecommand{\system}{\overset{\mathcal{H}}{ \longleftrightarrow}}
	%\newcommand{\solution}[2]{\textbf{Solution:}{#1}}
\newcommand{\solution}{\noindent \textbf{Solution: }}
\newcommand{\cosec}{\,\text{cosec}\,}
\providecommand{\dec}[2]{\ensuremath{\overset{#1}{\underset{#2}{\gtrless}}}}
\newcommand{\myvec}[1]{\ensuremath{\begin{pmatrix}#1\end{pmatrix}}}
\newcommand{\mydet}[1]{\ensuremath{\begin{vmatrix}#1\end{vmatrix}}}
%\numberwithin{equation}{section}
\numberwithin{equation}{subsection}
%\numberwithin{problem}{section}
%\numberwithin{definition}{section}
\makeatletter
\@addtoreset{figure}{problem}
\makeatother

\let\StandardTheFigure\thefigure
\let\vec\mathbf
%\renewcommand{\thefigure}{\theproblem.\arabic{figure}}
\renewcommand{\thefigure}{\theproblem}
%\setlist[enumerate,1]{before=\renewcommand\theequation{\theenumi.\arabic{equation}}
%\counterwithin{equation}{enumi}


%\renewcommand{\theequation}{\arabic{subsection}.\arabic{equation}}

\def\putbox#1#2#3{\makebox[0in][l]{\makebox[#1][l]{}\raisebox{\baselineskip}[0in][0in]{\raisebox{#2}[0in][0in]{#3}}}}
     \def\rightbox#1{\makebox[0in][r]{#1}}
     \def\centbox#1{\makebox[0in]{#1}}
     \def\topbox#1{\raisebox{-\baselineskip}[0in][0in]{#1}}
     \def\midbox#1{\raisebox{-0.5\baselineskip}[0in][0in]{#1}}

\vspace{3cm}


\title{Quiz 4}
\author{S Nithish}





% make the title area
\maketitle

\newpage

%\tableofcontents

\bigskip

\renewcommand{\thefigure}{\theenumi}
\renewcommand{\thetable}{\theenumi}
%\renewcommand{\theequation}{\theenumi}


\begin{abstract}
This document contains the solution of the question from NCERT 11th standard chapter 10 exercise 10.1 problem 7
\end{abstract}

%Download all python codes 
%
%\begin{lstlisting}
%svn co https://github.com/JayatiD93/trunk/My_solution_design/codes
%\end{lstlisting}

%Download all and latex-tikz codes from 
%
%\begin{lstlisting}
%svn co https://github.com/gadepall/school/trunk/ncert/geometry/figs
%\end{lstlisting}
%


\section{Exercise 10.1}

\begin{enumerate}

	\fi
Let the direction vector of the y-axis be
\begin{align}
\vec{m_1} = \myvec{0 \\ 1}
\end{align}
and the direction vector of the line be,
\begin{align}
\vec{m_2} = \myvec{1 \\ m}
\end{align}
where $m$ is the slope of the line.
Then, 
\begin{align}
	\vec{m_1}^{\top} \vec{m_2}   = m, \,
	\norm{\vec{m_1}}   = 1, \,
	\norm{\vec{m_2}}  = \sqrt{1+m^2}
\end{align}
yielding
the angle between the two as
\begin{align}
	\cos (\phi) &= \frac{m}{\sqrt{1+m^2}}
	 = \frac{\sqrt{3}}{2}\\
\implies
	m &= \pm \sqrt{3} 
\end{align}
Thus, 
$m=\sqrt{3}$ is the correct slope.


\item  Show that the line through the points $(1,-1,2),(3,4,-2 )$ is perpendicular to the line through the points$(0,3,2)$ and$(3,5,6)$.\\
    \solution
		\iffalse
\documentclass[journal,12pt,twocolumn]{IEEEtran}
%
\usepackage{setspace}
\usepackage{gensymb}
%\doublespacing
\singlespacing

%\usepackage{graphicx}
%\usepackage{amssymb}
%\usepackage{relsize}
\usepackage[cmex10]{amsmath}
%\usepackage{amsthm}
%\interdisplaylinepenalty=2500
%\savesymbol{iint}
%\usepackage{txfonts}
%\restoresymbol{TXF}{iint}
%\usepackage{wasysym}
\usepackage{amsthm}
%\usepackage{iithtlc}
\usepackage{mathrsfs}
\usepackage{txfonts}
\usepackage{stfloats}
\usepackage{bm}
\usepackage{cite}
\usepackage{cases}
\usepackage{subfig}
%\usepackage{xtab}
\usepackage{longtable}
\usepackage{multirow}
%\usepackage{algorithm}
%\usepackage{algpseudocode}
\usepackage{enumitem}
\usepackage{mathtools}
\usepackage{steinmetz}
\usepackage{tikz}
\usepackage{circuitikz}
\usepackage{verbatim}
\usepackage{tfrupee}
\usepackage[breaklinks=true]{hyperref}
%\usepackage{stmaryrd}
\usepackage{tkz-euclide} % loads  TikZ and tkz-base
%\usetkzobj{all}
\usetikzlibrary{calc,math}
\usepackage{listings}
    \usepackage{color}                                            %%
    \usepackage{array}                                            %%
    \usepackage{longtable}                                        %%
    \usepackage{calc}                                             %%
    \usepackage{multirow}                                         %%
    \usepackage{hhline}                                           %%
    \usepackage{ifthen}                                           %%
  %optionally (for landscape tables embedded in another document): %%
    \usepackage{lscape}     
\usepackage{multicol}
\usepackage{chngcntr}
%\usepackage{enumerate}

%\usepackage{wasysym}
%\newcounter{MYtempeqncnt}
\DeclareMathOperator*{\Res}{Res}
%\renewcommand{\baselinestretch}{2}
\renewcommand\thesection{\arabic{section}}
\renewcommand\thesubsection{\thesection.\arabic{subsection}}
\renewcommand\thesubsubsection{\thesubsection.\arabic{subsubsection}}

\renewcommand\thesectiondis{\arabic{section}}
\renewcommand\thesubsectiondis{\thesectiondis.\arabic{subsection}}
\renewcommand\thesubsubsectiondis{\thesubsectiondis.\arabic{subsubsection}}

% correct bad hyphenation here
\hyphenation{op-tical net-works semi-conduc-tor}
\def\inputGnumericTable{}                                 %%

\lstset{
%language=C,
frame=single, 
breaklines=true,
columns=fullflexible
}
%\lstset{
%language=tex,
%frame=single, 
%breaklines=true
%}


\begin{document}
%


\newtheorem{theorem}{Theorem}[section]
\newtheorem{problem}{Problem}
\newtheorem{proposition}{Proposition}[section]
\newtheorem{lemma}{Lemma}[section]
\newtheorem{corollary}[theorem]{Corollary}
\newtheorem{example}{Example}[section]
\newtheorem{definition}[problem]{Definition}
%\newtheorem{thm}{Theorem}[section] 
%\newtheorem{defn}[thm]{Definition}
%\newtheorem{algorithm}{Algorithm}[section]
%\newtheorem{cor}{Corollary}
\newcommand{\BEQA}{\begin{eqnarray}}
\newcommand{\EEQA}{\end{eqnarray}}
\newcommand{\define}{\stackrel{\triangle}{=}}

\bibliographystyle{IEEEtran}
%\bibliographystyle{ieeetr}


\providecommand{\mbf}{\mathbf}
\providecommand{\pr}[1]{\ensuremath{\Pr\left(#1\right)}}
\providecommand{\qfunc}[1]{\ensuremath{Q\left(#1\right)}}
\providecommand{\sbrak}[1]{\ensuremath{{}\left[#1\right]}}
\providecommand{\lsbrak}[1]{\ensuremath{{}\left[#1\right.}}
\providecommand{\rsbrak}[1]{\ensuremath{{}\left.#1\right]}}
\providecommand{\brak}[1]{\ensuremath{\left(#1\right)}}
\providecommand{\lbrak}[1]{\ensuremath{\left(#1\right.}}
\providecommand{\rbrak}[1]{\ensuremath{\left.#1\right)}}
\providecommand{\cbrak}[1]{\ensuremath{\left\{#1\right\}}}
\providecommand{\lcbrak}[1]{\ensuremath{\left\{#1\right.}}
\providecommand{\rcbrak}[1]{\ensuremath{\left.#1\right\}}}
\theoremstyle{remark}
\newtheorem{rem}{Remark}
\newcommand{\sgn}{\mathop{\mathrm{sgn}}}
\providecommand{\abs}[1]{\left\vert#1\right\vert}
\providecommand{\res}[1]{\Res\displaylimits_{#1}} 
\providecommand{\norm}[1]{\left\lVert#1\right\rVert}
%\providecommand{\norm}[1]{\lVert#1\rVert}
\providecommand{\mtx}[1]{\mathbf{#1}}
\providecommand{\mean}[1]{E\left[ #1 \right]}
\providecommand{\fourier}{\overset{\mathcal{F}}{ \rightleftharpoons}}
%\providecommand{\hilbert}{\overset{\mathcal{H}}{ \rightleftharpoons}}
\providecommand{\system}{\overset{\mathcal{H}}{ \longleftrightarrow}}
	%\newcommand{\solution}[2]{\textbf{Solution:}{#1}}
\newcommand{\solution}{\noindent \textbf{Solution: }}
\newcommand{\cosec}{\,\text{cosec}\,}
\providecommand{\dec}[2]{\ensuremath{\overset{#1}{\underset{#2}{\gtrless}}}}
\newcommand{\myvec}[1]{\ensuremath{\begin{pmatrix}#1\end{pmatrix}}}
\newcommand{\mydet}[1]{\ensuremath{\begin{vmatrix}#1\end{vmatrix}}}
%\numberwithin{equation}{section}
\numberwithin{equation}{subsection}
%\numberwithin{problem}{section}
%\numberwithin{definition}{section}
\makeatletter
\@addtoreset{figure}{problem}
\makeatother

\let\StandardTheFigure\thefigure
\let\vec\mathbf
%\renewcommand{\thefigure}{\theproblem.\arabic{figure}}
\renewcommand{\thefigure}{\theproblem}
%\setlist[enumerate,1]{before=\renewcommand\theequation{\theenumi.\arabic{equation}}
%\counterwithin{equation}{enumi}


%\renewcommand{\theequation}{\arabic{subsection}.\arabic{equation}}

\def\putbox#1#2#3{\makebox[0in][l]{\makebox[#1][l]{}\raisebox{\baselineskip}[0in][0in]{\raisebox{#2}[0in][0in]{#3}}}}
     \def\rightbox#1{\makebox[0in][r]{#1}}
     \def\centbox#1{\makebox[0in]{#1}}
     \def\topbox#1{\raisebox{-\baselineskip}[0in][0in]{#1}}
     \def\midbox#1{\raisebox{-0.5\baselineskip}[0in][0in]{#1}}

\vspace{3cm}


\title{Quiz 4}
\author{S Nithish}





% make the title area
\maketitle

\newpage

%\tableofcontents

\bigskip

\renewcommand{\thefigure}{\theenumi}
\renewcommand{\thetable}{\theenumi}
%\renewcommand{\theequation}{\theenumi}


\begin{abstract}
This document contains the solution of the question from NCERT 11th standard chapter 10 exercise 10.1 problem 7
\end{abstract}

%Download all python codes 
%
%\begin{lstlisting}
%svn co https://github.com/JayatiD93/trunk/My_solution_design/codes
%\end{lstlisting}

%Download all and latex-tikz codes from 
%
%\begin{lstlisting}
%svn co https://github.com/gadepall/school/trunk/ncert/geometry/figs
%\end{lstlisting}
%


\section{Exercise 10.1}

\begin{enumerate}

	\fi
Let the direction vector of the y-axis be
\begin{align}
\vec{m_1} = \myvec{0 \\ 1}
\end{align}
and the direction vector of the line be,
\begin{align}
\vec{m_2} = \myvec{1 \\ m}
\end{align}
where $m$ is the slope of the line.
Then, 
\begin{align}
	\vec{m_1}^{\top} \vec{m_2}   = m, \,
	\norm{\vec{m_1}}   = 1, \,
	\norm{\vec{m_2}}  = \sqrt{1+m^2}
\end{align}
yielding
the angle between the two as
\begin{align}
	\cos (\phi) &= \frac{m}{\sqrt{1+m^2}}
	 = \frac{\sqrt{3}}{2}\\
\implies
	m &= \pm \sqrt{3} 
\end{align}
Thus, 
$m=\sqrt{3}$ is the correct slope.


\item Show that the line through the points $(4,7,8),(2,3,4)$ is parallel to the line through the points $(-1,-2,1),(1,2,5)$.\\
\item  Find the equation of the line which passes through the point $(1,2,3)$ and is parallel to the vector $3\hat{i}+2\hat{j}-2\hat{k}$\\
\item  Find the equation of the line in vector and in cartesian form that passes through the point with position vector $2\hat{i}-\hat{j}+4\hat{k}$ and is in direction $\hat{i}+2\hat{j}-\hat{k}$.\\
    \solution
		\begin{align}
	\myvec{3&-4}\vec{x}=\myvec{3&-4}\myvec{-2\\3}
	=-18 
\end{align}
is the required equation of the line.

\item Find the cartesian equation of the line which passes through the point $(-2,4,-5)$ and parallel to the line given by$ \frac{x+3}{3}=\frac{y-4}{5}=\frac{z+8}{6}$.\\
\item The cartesian equation of a line is $ \frac{x-5}{3}=\frac{y+4}{7}=\frac{z-6}{2}$. Write its vector form.\\
\item Find the vector and the cartesian equations of the lines that passes through the origin and $(5,-2,3)$.\\
    \solution
		\begin{align}
	\myvec{3&-4}\vec{x}=\myvec{3&-4}\myvec{-2\\3}
	=-18 
\end{align}
is the required equation of the line.

\item Find the vector and the cartesian equations of the line that passes through the points $(3,-2,-5),(3,-2,6)$.\\
    \solution
		\begin{align}
	\myvec{3&-4}\vec{x}=\myvec{3&-4}\myvec{-2\\3}
	=-18 
\end{align}
is the required equation of the line.

\item  Find the angle between the following pairs of lines:
\begin{enumerate}
\item  $\overrightarrow{r}=2\hat{i}-5\hat{j}+\hat{k}+\lambda(3\hat{i}+2\hat{j}+6\hat{k})$ and\\ $\overrightarrow{r}=7\hat{i}-6\hat{k}+\mu(\hat{i}+2\hat{j}+2\hat{k})$
\item   $\overrightarrow{r}=3\hat{i}+\hat{j}-2\hat{k}+\lambda(\hat{i}-\hat{j}-2\hat{k})$ and\\ $\overrightarrow{r}=2\hat{i}-\hat{j}-56\hat{k}+\mu(3\hat{i}-5\hat{j}-4\hat{k})$ 
\end{enumerate}
\item Find the angle between the following pairs of lines:
\begin{enumerate}
\item $ \frac{x-2}{2}=\frac{y-1}{5}=\frac{z+3}{-3}$ and $ \frac{x+2}{-1}=\frac{y-4}{8}=\frac{z-5}{4}$.
\item $ \frac{x}{2}=\frac{y}{2}=\frac{z}{1}$ and $ \frac{x-5}{4}=\frac{y-2}{1}=\frac{z-3}{8}$.
\end{enumerate}
\item Find the values of p so that the lines $ \frac{1-x}{3}=\frac{7y-14}{2p}=\frac{z-3}{2}$ and $ \frac{7-7x}{3p}=\frac{y-5}{1}=\frac{6-z}{5}$ are at right angles.
\item Show that the lines $ \frac{x-5}{7}=\frac{y+2}{-5}=\frac{z}{1}$ and $ \frac{x}{1}=\frac{y}{2}=\frac{z}{3}$ are perpendicular to each other.
	\\
    \solution
		\iffalse
\documentclass[journal,12pt,twocolumn]{IEEEtran}
%
\usepackage{setspace}
\usepackage{gensymb}
%\doublespacing
\singlespacing

%\usepackage{graphicx}
%\usepackage{amssymb}
%\usepackage{relsize}
\usepackage[cmex10]{amsmath}
%\usepackage{amsthm}
%\interdisplaylinepenalty=2500
%\savesymbol{iint}
%\usepackage{txfonts}
%\restoresymbol{TXF}{iint}
%\usepackage{wasysym}
\usepackage{amsthm}
%\usepackage{iithtlc}
\usepackage{mathrsfs}
\usepackage{txfonts}
\usepackage{stfloats}
\usepackage{bm}
\usepackage{cite}
\usepackage{cases}
\usepackage{subfig}
%\usepackage{xtab}
\usepackage{longtable}
\usepackage{multirow}
%\usepackage{algorithm}
%\usepackage{algpseudocode}
\usepackage{enumitem}
\usepackage{mathtools}
\usepackage{steinmetz}
\usepackage{tikz}
\usepackage{circuitikz}
\usepackage{verbatim}
\usepackage{tfrupee}
\usepackage[breaklinks=true]{hyperref}
%\usepackage{stmaryrd}
\usepackage{tkz-euclide} % loads  TikZ and tkz-base
%\usetkzobj{all}
\usetikzlibrary{calc,math}
\usepackage{listings}
    \usepackage{color}                                            %%
    \usepackage{array}                                            %%
    \usepackage{longtable}                                        %%
    \usepackage{calc}                                             %%
    \usepackage{multirow}                                         %%
    \usepackage{hhline}                                           %%
    \usepackage{ifthen}                                           %%
  %optionally (for landscape tables embedded in another document): %%
    \usepackage{lscape}     
\usepackage{multicol}
\usepackage{chngcntr}
%\usepackage{enumerate}

%\usepackage{wasysym}
%\newcounter{MYtempeqncnt}
\DeclareMathOperator*{\Res}{Res}
%\renewcommand{\baselinestretch}{2}
\renewcommand\thesection{\arabic{section}}
\renewcommand\thesubsection{\thesection.\arabic{subsection}}
\renewcommand\thesubsubsection{\thesubsection.\arabic{subsubsection}}

\renewcommand\thesectiondis{\arabic{section}}
\renewcommand\thesubsectiondis{\thesectiondis.\arabic{subsection}}
\renewcommand\thesubsubsectiondis{\thesubsectiondis.\arabic{subsubsection}}

% correct bad hyphenation here
\hyphenation{op-tical net-works semi-conduc-tor}
\def\inputGnumericTable{}                                 %%

\lstset{
%language=C,
frame=single, 
breaklines=true,
columns=fullflexible
}
%\lstset{
%language=tex,
%frame=single, 
%breaklines=true
%}


\begin{document}
%


\newtheorem{theorem}{Theorem}[section]
\newtheorem{problem}{Problem}
\newtheorem{proposition}{Proposition}[section]
\newtheorem{lemma}{Lemma}[section]
\newtheorem{corollary}[theorem]{Corollary}
\newtheorem{example}{Example}[section]
\newtheorem{definition}[problem]{Definition}
%\newtheorem{thm}{Theorem}[section] 
%\newtheorem{defn}[thm]{Definition}
%\newtheorem{algorithm}{Algorithm}[section]
%\newtheorem{cor}{Corollary}
\newcommand{\BEQA}{\begin{eqnarray}}
\newcommand{\EEQA}{\end{eqnarray}}
\newcommand{\define}{\stackrel{\triangle}{=}}

\bibliographystyle{IEEEtran}
%\bibliographystyle{ieeetr}


\providecommand{\mbf}{\mathbf}
\providecommand{\pr}[1]{\ensuremath{\Pr\left(#1\right)}}
\providecommand{\qfunc}[1]{\ensuremath{Q\left(#1\right)}}
\providecommand{\sbrak}[1]{\ensuremath{{}\left[#1\right]}}
\providecommand{\lsbrak}[1]{\ensuremath{{}\left[#1\right.}}
\providecommand{\rsbrak}[1]{\ensuremath{{}\left.#1\right]}}
\providecommand{\brak}[1]{\ensuremath{\left(#1\right)}}
\providecommand{\lbrak}[1]{\ensuremath{\left(#1\right.}}
\providecommand{\rbrak}[1]{\ensuremath{\left.#1\right)}}
\providecommand{\cbrak}[1]{\ensuremath{\left\{#1\right\}}}
\providecommand{\lcbrak}[1]{\ensuremath{\left\{#1\right.}}
\providecommand{\rcbrak}[1]{\ensuremath{\left.#1\right\}}}
\theoremstyle{remark}
\newtheorem{rem}{Remark}
\newcommand{\sgn}{\mathop{\mathrm{sgn}}}
\providecommand{\abs}[1]{\left\vert#1\right\vert}
\providecommand{\res}[1]{\Res\displaylimits_{#1}} 
\providecommand{\norm}[1]{\left\lVert#1\right\rVert}
%\providecommand{\norm}[1]{\lVert#1\rVert}
\providecommand{\mtx}[1]{\mathbf{#1}}
\providecommand{\mean}[1]{E\left[ #1 \right]}
\providecommand{\fourier}{\overset{\mathcal{F}}{ \rightleftharpoons}}
%\providecommand{\hilbert}{\overset{\mathcal{H}}{ \rightleftharpoons}}
\providecommand{\system}{\overset{\mathcal{H}}{ \longleftrightarrow}}
	%\newcommand{\solution}[2]{\textbf{Solution:}{#1}}
\newcommand{\solution}{\noindent \textbf{Solution: }}
\newcommand{\cosec}{\,\text{cosec}\,}
\providecommand{\dec}[2]{\ensuremath{\overset{#1}{\underset{#2}{\gtrless}}}}
\newcommand{\myvec}[1]{\ensuremath{\begin{pmatrix}#1\end{pmatrix}}}
\newcommand{\mydet}[1]{\ensuremath{\begin{vmatrix}#1\end{vmatrix}}}
%\numberwithin{equation}{section}
\numberwithin{equation}{subsection}
%\numberwithin{problem}{section}
%\numberwithin{definition}{section}
\makeatletter
\@addtoreset{figure}{problem}
\makeatother

\let\StandardTheFigure\thefigure
\let\vec\mathbf
%\renewcommand{\thefigure}{\theproblem.\arabic{figure}}
\renewcommand{\thefigure}{\theproblem}
%\setlist[enumerate,1]{before=\renewcommand\theequation{\theenumi.\arabic{equation}}
%\counterwithin{equation}{enumi}


%\renewcommand{\theequation}{\arabic{subsection}.\arabic{equation}}

\def\putbox#1#2#3{\makebox[0in][l]{\makebox[#1][l]{}\raisebox{\baselineskip}[0in][0in]{\raisebox{#2}[0in][0in]{#3}}}}
     \def\rightbox#1{\makebox[0in][r]{#1}}
     \def\centbox#1{\makebox[0in]{#1}}
     \def\topbox#1{\raisebox{-\baselineskip}[0in][0in]{#1}}
     \def\midbox#1{\raisebox{-0.5\baselineskip}[0in][0in]{#1}}

\vspace{3cm}


\title{Quiz 4}
\author{S Nithish}





% make the title area
\maketitle

\newpage

%\tableofcontents

\bigskip

\renewcommand{\thefigure}{\theenumi}
\renewcommand{\thetable}{\theenumi}
%\renewcommand{\theequation}{\theenumi}


\begin{abstract}
This document contains the solution of the question from NCERT 11th standard chapter 10 exercise 10.1 problem 7
\end{abstract}

%Download all python codes 
%
%\begin{lstlisting}
%svn co https://github.com/JayatiD93/trunk/My_solution_design/codes
%\end{lstlisting}

%Download all and latex-tikz codes from 
%
%\begin{lstlisting}
%svn co https://github.com/gadepall/school/trunk/ncert/geometry/figs
%\end{lstlisting}
%


\section{Exercise 10.1}

\begin{enumerate}

	\fi
Let the direction vector of the y-axis be
\begin{align}
\vec{m_1} = \myvec{0 \\ 1}
\end{align}
and the direction vector of the line be,
\begin{align}
\vec{m_2} = \myvec{1 \\ m}
\end{align}
where $m$ is the slope of the line.
Then, 
\begin{align}
	\vec{m_1}^{\top} \vec{m_2}   = m, \,
	\norm{\vec{m_1}}   = 1, \,
	\norm{\vec{m_2}}  = \sqrt{1+m^2}
\end{align}
yielding
the angle between the two as
\begin{align}
	\cos (\phi) &= \frac{m}{\sqrt{1+m^2}}
	 = \frac{\sqrt{3}}{2}\\
\implies
	m &= \pm \sqrt{3} 
\end{align}
Thus, 
$m=\sqrt{3}$ is the correct slope.


\item Find the shortest distance between the lines\\  $\overrightarrow{r}=(\hat{i}+2\hat{j}+\hat{k})+\lambda(\hat{i}-\hat{j}+\hat{k})$ and \\$\overrightarrow{r}=2\hat{i}-\hat{j}-\hat{k}+\mu(2\hat{i}+\hat{j}+2\hat{k})$
\item Find the shortest distance between the lines\\
$ \frac{x+1}{7}=\frac{y+1}{-6}=\frac{z+1}{1}$ and $ \frac{x-3}{1}=\frac{y-5}{-2}=\frac{z-7}{1}$ 
    \solution
		%\begin{enumerate}[label=\arabic*.,ref=\theenumi]
\begin{enumerate}[label=\thesubsection.\arabic*.,ref=\thesubsection.\theenumi]
	\item The lines
\begin{align}
\begin{split}
	L_1: \quad   \vec{x} &=\vec{A}+ \kappa_1\vec{m_1}
	\\
L_2: \quad        
	\vec{x} &= \vec{B}  + \kappa_2\vec{m_2} 
\end{split}
	    \label{eq:chapters/12/11/2/16/lsq/L1L2}
\end{align}
will intersect if 
\begin{align}
\vec{A}+ \kappa_1\vec{m_1}
= \vec{B}  + \kappa_2\vec{m_2} 
\\
\implies 
 \myvec{\vec{m_1} & \vec{m_2}}\myvec{\kappa_1\\-\kappa_2}
	 =\vec{B}-\vec{A}
 \\
	\implies \rank\myvec{\vec{M}  
	& \vec{B}-\vec{A}} = 2 
	    \label{eq:chapters/12/11/2/16/lsq/rank}
\end{align}
where
\begin{align}
	\vec{M} = 
	\myvec{\vec{m_1} & \vec{m_2}} 
\end{align}
\item If $L_1, L_2$, do not intersect, let 
\begin{align}
\begin{split}
	\vec{x}_1 &=\vec{A}+ \kappa_1\vec{m_1}
	\\
	\vec{x}_2 &= \vec{B}  + \kappa_2\vec{m_2} 
\end{split}
	    \label{eq:chapters/12/11/2/16/lsq/x1x2}
\end{align}
be points on 
$L_1, L_2$ respectively, that are closest to each other.
Then, 
	    from \eqref{eq:chapters/12/11/2/16/lsq/x1x2}
\begin{align}
\vec{x_1} - \vec{x_2} =
	 \vec{A}-\vec{B}+
 \myvec{\vec{m_1} & \vec{m_2}}\myvec{\kappa_1\\-\kappa_2}
	\label{eq:chapters/12/11/2/16/lsq/x-diff}
\end{align}
Also, 
    \begin{align}
	    \brak{\vec{x}_1 -\vec{x}_2}^\top\vec{m}_1
	    =
	    \brak{\vec{x}_1 -\vec{x}_2}^\top\vec{m}_2
	    =0
	    \\
	    \implies 
	    \brak{\vec{x}_1 -\vec{x}_2}^\top\myvec{\vec{m_1} & \vec{m_2}} = \vec{0}
	    \\
	    \text{or, }	    \vec{M}^\top\brak{\vec{x}_1 -\vec{x}_2} = \vec{0}
	    \\
	    \implies \vec{M}^\top
	    \brak{\vec{A}-\vec{B}}+
 \vec{M}^\top\vec{M}\myvec{\kappa_1\\-\kappa_2} = \vec{0}
	    \label{eq:chapters/12/11/2/16/lsq/m-orth}
    \end{align}
	    from 
	\eqref{eq:chapters/12/11/2/16/lsq/x-diff},
	yielding
    \begin{align}
	    \vec{M}^\top\vec{M}\myvec{\kappa_1\\-\kappa_2} = \vec{M}^\top\brak{\vec{B}-\vec{A}}
        \label{eq:chapters/12/11/2/16/lsq/vec-eqn}
    \end{align}
    This is known as the {\em least squares solution}.
	\item Perform the eigendecompositions 
    \begin{align}
	    \vec{MM}^\top &= \vec{U}\vec{D}_1\vec{U}^\top \label{eq:chapters/12/11/2/16/svd/decomp-1} \\
	    \vec{M}^\top\vec{M} &= \vec{V}\vec{D}_2\vec{V}^\top \label{eq:chapters/12/11/2/16/svd/decomp-2}
    \end{align}
	\item    The following expression is known as {\em singular value decomposition}
    \begin{align}
        \vec{M} = 
	\vec{U}\vec{\Sigma}\vec{V}^\top
        \label{eq:chapters/12/11/2/16/svd/M-svd}
    \end{align}
    where $\vec{\Sigma}$ is diagonal with
    entries obtained as in 
        \eqref{eq:chapters/12/11/2/16/svd/svd-params}.
 Substituting in 
        \eqref{eq:chapters/12/11/2/16/lsq/vec-eqn},
	\begin{align}
\vec{V}\vec{\Sigma}\vec{U}^\top\vec{U}\vec{\Sigma}\vec{V}^\top\bm{\kappa} &= \vec{V}\vec{\Sigma}\vec{U}^\top\brak{\vec{B}-\vec{A}} \\
\implies \vec{V}\vec{\Sigma}^2\vec{V}^\top\bm{\kappa} &= \vec{V}\vec{\Sigma}\vec{U}^\top\brak{\vec{B}-\vec{A}} \\
\implies \bm{\kappa} &= \brak{\vec{V}\vec{\Sigma}^2\vec{V}^\top}^{-1}\vec{V}\vec{\Sigma}\vec{U}^\top\brak{\vec{B}-\vec{A}} \\
\implies \bm{\kappa} &= \vec{V}\vec{\Sigma}^{-2}\vec{V}^\top\vec{V}\vec{\Sigma}\vec{U}^\top\brak{\vec{B}-\vec{A}} \\
\implies \bm{\kappa} &= \vec{V}\vec{\Sigma}^{-1}\vec{U}^\top\brak{\vec{B}-\vec{A}}
\label{eq:chapters/12/11/2/16/svd/kappa-sol}
\end{align}
    where $\vec{\Sigma}^{-1}$ is obtained by inverting the nonzero elements of
    $\vec{\Sigma}$. 
		\item 
	    From \eqref{eq:chapters/12/11/2/16/lsq/x1x2}, 
\begin{align}
	\vec{x}_1-\vec{x}_2 &= 
	\vec{A}+ \kappa_1\vec{m_1}
	 -\vec{B}  - \kappa_2\vec{m_2} 
	 \\
	 &=
	\vec{A} 
	 -\vec{B}  + \vec{M} 
	\bm{\kappa}
\end{align}
which, upon substitution from 
        \eqref{eq:chapters/12/11/2/16/svd/M-svd}
	yields
\begin{align}
	\vec{x}_1-\vec{x}_2 &= 
	\vec{A} 
	 -\vec{B}  + 
\vec{U}\vec{\Sigma}\vec{V}^\top
\vec{V}\vec{\Sigma}^{-1}\vec{U}^\top\brak{\vec{B}-\vec{A}}
\\
	&=
	\brak{	\vec{A} 
	 -\vec{B}}  \brak{\vec{I}- 
\vec{U}\vec{\Sigma}
	\vec{\Sigma}^{-1}\vec{U}^\top}
\end{align}
			Thus, 
    \begin{align}
	    \norm{\vec{x}_1-\vec{x}_2} = 
	\norm{\brak{	\vec{A} 
	 -\vec{B}}  \brak{\vec{I}- 
\vec{U}\vec{\Sigma}
	    \vec{\Sigma}^{-1}\vec{U}^\top}}
        \label{eq:chapters/12/11/2/16/svd/min-sol}
    \end{align}
\item Least squares solution
	\begin{lstlisting}
	codes/book/skew_least.py
\end{lstlisting}
\item Least squares using builtin SVD 
	\begin{lstlisting}
	codes/book/skew_builtin.py
\end{lstlisting}
\item Code linking eigenvalues and singular values
	\begin{lstlisting}
	codes/book/skew_svd.py
\end{lstlisting}
\end{enumerate}

    \item Find the shortest distance between the lines whose vector equations are
    \begin{align}
        \vec{x} = \myvec{1\\2\\3} + \lambda_1\myvec{1\\-3\\2}
        \label{eq:chapters/12/11/2/16/L1}
    \end{align}
    and
    \begin{align}
        \vec{x} = \myvec{4\\5\\6} + \lambda_2\myvec{2\\3\\1}
        \label{eq:chapters/12/11/2/16/L2}
    \end{align}
    \solution
		%\begin{enumerate}[label=\arabic*.,ref=\theenumi]
\begin{enumerate}[label=\thesubsection.\arabic*.,ref=\thesubsection.\theenumi]
	\item The lines
\begin{align}
\begin{split}
	L_1: \quad   \vec{x} &=\vec{A}+ \kappa_1\vec{m_1}
	\\
L_2: \quad        
	\vec{x} &= \vec{B}  + \kappa_2\vec{m_2} 
\end{split}
	    \label{eq:chapters/12/11/2/16/lsq/L1L2}
\end{align}
will intersect if 
\begin{align}
\vec{A}+ \kappa_1\vec{m_1}
= \vec{B}  + \kappa_2\vec{m_2} 
\\
\implies 
 \myvec{\vec{m_1} & \vec{m_2}}\myvec{\kappa_1\\-\kappa_2}
	 =\vec{B}-\vec{A}
 \\
	\implies \rank\myvec{\vec{M}  
	& \vec{B}-\vec{A}} = 2 
	    \label{eq:chapters/12/11/2/16/lsq/rank}
\end{align}
where
\begin{align}
	\vec{M} = 
	\myvec{\vec{m_1} & \vec{m_2}} 
\end{align}
\item If $L_1, L_2$, do not intersect, let 
\begin{align}
\begin{split}
	\vec{x}_1 &=\vec{A}+ \kappa_1\vec{m_1}
	\\
	\vec{x}_2 &= \vec{B}  + \kappa_2\vec{m_2} 
\end{split}
	    \label{eq:chapters/12/11/2/16/lsq/x1x2}
\end{align}
be points on 
$L_1, L_2$ respectively, that are closest to each other.
Then, 
	    from \eqref{eq:chapters/12/11/2/16/lsq/x1x2}
\begin{align}
\vec{x_1} - \vec{x_2} =
	 \vec{A}-\vec{B}+
 \myvec{\vec{m_1} & \vec{m_2}}\myvec{\kappa_1\\-\kappa_2}
	\label{eq:chapters/12/11/2/16/lsq/x-diff}
\end{align}
Also, 
    \begin{align}
	    \brak{\vec{x}_1 -\vec{x}_2}^\top\vec{m}_1
	    =
	    \brak{\vec{x}_1 -\vec{x}_2}^\top\vec{m}_2
	    =0
	    \\
	    \implies 
	    \brak{\vec{x}_1 -\vec{x}_2}^\top\myvec{\vec{m_1} & \vec{m_2}} = \vec{0}
	    \\
	    \text{or, }	    \vec{M}^\top\brak{\vec{x}_1 -\vec{x}_2} = \vec{0}
	    \\
	    \implies \vec{M}^\top
	    \brak{\vec{A}-\vec{B}}+
 \vec{M}^\top\vec{M}\myvec{\kappa_1\\-\kappa_2} = \vec{0}
	    \label{eq:chapters/12/11/2/16/lsq/m-orth}
    \end{align}
	    from 
	\eqref{eq:chapters/12/11/2/16/lsq/x-diff},
	yielding
    \begin{align}
	    \vec{M}^\top\vec{M}\myvec{\kappa_1\\-\kappa_2} = \vec{M}^\top\brak{\vec{B}-\vec{A}}
        \label{eq:chapters/12/11/2/16/lsq/vec-eqn}
    \end{align}
    This is known as the {\em least squares solution}.
	\item Perform the eigendecompositions 
    \begin{align}
	    \vec{MM}^\top &= \vec{U}\vec{D}_1\vec{U}^\top \label{eq:chapters/12/11/2/16/svd/decomp-1} \\
	    \vec{M}^\top\vec{M} &= \vec{V}\vec{D}_2\vec{V}^\top \label{eq:chapters/12/11/2/16/svd/decomp-2}
    \end{align}
	\item    The following expression is known as {\em singular value decomposition}
    \begin{align}
        \vec{M} = 
	\vec{U}\vec{\Sigma}\vec{V}^\top
        \label{eq:chapters/12/11/2/16/svd/M-svd}
    \end{align}
    where $\vec{\Sigma}$ is diagonal with
    entries obtained as in 
        \eqref{eq:chapters/12/11/2/16/svd/svd-params}.
 Substituting in 
        \eqref{eq:chapters/12/11/2/16/lsq/vec-eqn},
	\begin{align}
\vec{V}\vec{\Sigma}\vec{U}^\top\vec{U}\vec{\Sigma}\vec{V}^\top\bm{\kappa} &= \vec{V}\vec{\Sigma}\vec{U}^\top\brak{\vec{B}-\vec{A}} \\
\implies \vec{V}\vec{\Sigma}^2\vec{V}^\top\bm{\kappa} &= \vec{V}\vec{\Sigma}\vec{U}^\top\brak{\vec{B}-\vec{A}} \\
\implies \bm{\kappa} &= \brak{\vec{V}\vec{\Sigma}^2\vec{V}^\top}^{-1}\vec{V}\vec{\Sigma}\vec{U}^\top\brak{\vec{B}-\vec{A}} \\
\implies \bm{\kappa} &= \vec{V}\vec{\Sigma}^{-2}\vec{V}^\top\vec{V}\vec{\Sigma}\vec{U}^\top\brak{\vec{B}-\vec{A}} \\
\implies \bm{\kappa} &= \vec{V}\vec{\Sigma}^{-1}\vec{U}^\top\brak{\vec{B}-\vec{A}}
\label{eq:chapters/12/11/2/16/svd/kappa-sol}
\end{align}
    where $\vec{\Sigma}^{-1}$ is obtained by inverting the nonzero elements of
    $\vec{\Sigma}$. 
		\item 
	    From \eqref{eq:chapters/12/11/2/16/lsq/x1x2}, 
\begin{align}
	\vec{x}_1-\vec{x}_2 &= 
	\vec{A}+ \kappa_1\vec{m_1}
	 -\vec{B}  - \kappa_2\vec{m_2} 
	 \\
	 &=
	\vec{A} 
	 -\vec{B}  + \vec{M} 
	\bm{\kappa}
\end{align}
which, upon substitution from 
        \eqref{eq:chapters/12/11/2/16/svd/M-svd}
	yields
\begin{align}
	\vec{x}_1-\vec{x}_2 &= 
	\vec{A} 
	 -\vec{B}  + 
\vec{U}\vec{\Sigma}\vec{V}^\top
\vec{V}\vec{\Sigma}^{-1}\vec{U}^\top\brak{\vec{B}-\vec{A}}
\\
	&=
	\brak{	\vec{A} 
	 -\vec{B}}  \brak{\vec{I}- 
\vec{U}\vec{\Sigma}
	\vec{\Sigma}^{-1}\vec{U}^\top}
\end{align}
			Thus, 
    \begin{align}
	    \norm{\vec{x}_1-\vec{x}_2} = 
	\norm{\brak{	\vec{A} 
	 -\vec{B}}  \brak{\vec{I}- 
\vec{U}\vec{\Sigma}
	    \vec{\Sigma}^{-1}\vec{U}^\top}}
        \label{eq:chapters/12/11/2/16/svd/min-sol}
    \end{align}
\item Least squares solution
	\begin{lstlisting}
	codes/book/skew_least.py
\end{lstlisting}
\item Least squares using builtin SVD 
	\begin{lstlisting}
	codes/book/skew_builtin.py
\end{lstlisting}
\item Code linking eigenvalues and singular values
	\begin{lstlisting}
	codes/book/skew_svd.py
\end{lstlisting}
\end{enumerate}

\item Find the shortest distance between the lines whose vector equations are \\
 $\overrightarrow{r}=(1-t)\hat{i}+(t-2)\hat{j}+(3-2t)\hat{k}$     and  \\$\overrightarrow{r}=(s+1)\hat{i}+(2s-1)\hat{j}-(2s+1)\hat{k}$
 \item 
\item Find the shortest distance between the lines $l_1$ and $l_2$ whose vector equations are ${\overrightarrow{r} = \hat{i}+\hat{j}+\lambda(2\hat{i}-\hat{j}+\hat{k})}$ and ${\overrightarrow{r} = 2\hat{i}+\hat{j}-\hat{k}+\mu(3\hat{i}-5\hat{j}+2\hat{k})}$.
    \solution
		%\begin{enumerate}[label=\arabic*.,ref=\theenumi]
\begin{enumerate}[label=\thesubsection.\arabic*.,ref=\thesubsection.\theenumi]
	\item The lines
\begin{align}
\begin{split}
	L_1: \quad   \vec{x} &=\vec{A}+ \kappa_1\vec{m_1}
	\\
L_2: \quad        
	\vec{x} &= \vec{B}  + \kappa_2\vec{m_2} 
\end{split}
	    \label{eq:chapters/12/11/2/16/lsq/L1L2}
\end{align}
will intersect if 
\begin{align}
\vec{A}+ \kappa_1\vec{m_1}
= \vec{B}  + \kappa_2\vec{m_2} 
\\
\implies 
 \myvec{\vec{m_1} & \vec{m_2}}\myvec{\kappa_1\\-\kappa_2}
	 =\vec{B}-\vec{A}
 \\
	\implies \rank\myvec{\vec{M}  
	& \vec{B}-\vec{A}} = 2 
	    \label{eq:chapters/12/11/2/16/lsq/rank}
\end{align}
where
\begin{align}
	\vec{M} = 
	\myvec{\vec{m_1} & \vec{m_2}} 
\end{align}
\item If $L_1, L_2$, do not intersect, let 
\begin{align}
\begin{split}
	\vec{x}_1 &=\vec{A}+ \kappa_1\vec{m_1}
	\\
	\vec{x}_2 &= \vec{B}  + \kappa_2\vec{m_2} 
\end{split}
	    \label{eq:chapters/12/11/2/16/lsq/x1x2}
\end{align}
be points on 
$L_1, L_2$ respectively, that are closest to each other.
Then, 
	    from \eqref{eq:chapters/12/11/2/16/lsq/x1x2}
\begin{align}
\vec{x_1} - \vec{x_2} =
	 \vec{A}-\vec{B}+
 \myvec{\vec{m_1} & \vec{m_2}}\myvec{\kappa_1\\-\kappa_2}
	\label{eq:chapters/12/11/2/16/lsq/x-diff}
\end{align}
Also, 
    \begin{align}
	    \brak{\vec{x}_1 -\vec{x}_2}^\top\vec{m}_1
	    =
	    \brak{\vec{x}_1 -\vec{x}_2}^\top\vec{m}_2
	    =0
	    \\
	    \implies 
	    \brak{\vec{x}_1 -\vec{x}_2}^\top\myvec{\vec{m_1} & \vec{m_2}} = \vec{0}
	    \\
	    \text{or, }	    \vec{M}^\top\brak{\vec{x}_1 -\vec{x}_2} = \vec{0}
	    \\
	    \implies \vec{M}^\top
	    \brak{\vec{A}-\vec{B}}+
 \vec{M}^\top\vec{M}\myvec{\kappa_1\\-\kappa_2} = \vec{0}
	    \label{eq:chapters/12/11/2/16/lsq/m-orth}
    \end{align}
	    from 
	\eqref{eq:chapters/12/11/2/16/lsq/x-diff},
	yielding
    \begin{align}
	    \vec{M}^\top\vec{M}\myvec{\kappa_1\\-\kappa_2} = \vec{M}^\top\brak{\vec{B}-\vec{A}}
        \label{eq:chapters/12/11/2/16/lsq/vec-eqn}
    \end{align}
    This is known as the {\em least squares solution}.
	\item Perform the eigendecompositions 
    \begin{align}
	    \vec{MM}^\top &= \vec{U}\vec{D}_1\vec{U}^\top \label{eq:chapters/12/11/2/16/svd/decomp-1} \\
	    \vec{M}^\top\vec{M} &= \vec{V}\vec{D}_2\vec{V}^\top \label{eq:chapters/12/11/2/16/svd/decomp-2}
    \end{align}
	\item    The following expression is known as {\em singular value decomposition}
    \begin{align}
        \vec{M} = 
	\vec{U}\vec{\Sigma}\vec{V}^\top
        \label{eq:chapters/12/11/2/16/svd/M-svd}
    \end{align}
    where $\vec{\Sigma}$ is diagonal with
    entries obtained as in 
        \eqref{eq:chapters/12/11/2/16/svd/svd-params}.
 Substituting in 
        \eqref{eq:chapters/12/11/2/16/lsq/vec-eqn},
	\begin{align}
\vec{V}\vec{\Sigma}\vec{U}^\top\vec{U}\vec{\Sigma}\vec{V}^\top\bm{\kappa} &= \vec{V}\vec{\Sigma}\vec{U}^\top\brak{\vec{B}-\vec{A}} \\
\implies \vec{V}\vec{\Sigma}^2\vec{V}^\top\bm{\kappa} &= \vec{V}\vec{\Sigma}\vec{U}^\top\brak{\vec{B}-\vec{A}} \\
\implies \bm{\kappa} &= \brak{\vec{V}\vec{\Sigma}^2\vec{V}^\top}^{-1}\vec{V}\vec{\Sigma}\vec{U}^\top\brak{\vec{B}-\vec{A}} \\
\implies \bm{\kappa} &= \vec{V}\vec{\Sigma}^{-2}\vec{V}^\top\vec{V}\vec{\Sigma}\vec{U}^\top\brak{\vec{B}-\vec{A}} \\
\implies \bm{\kappa} &= \vec{V}\vec{\Sigma}^{-1}\vec{U}^\top\brak{\vec{B}-\vec{A}}
\label{eq:chapters/12/11/2/16/svd/kappa-sol}
\end{align}
    where $\vec{\Sigma}^{-1}$ is obtained by inverting the nonzero elements of
    $\vec{\Sigma}$. 
		\item 
	    From \eqref{eq:chapters/12/11/2/16/lsq/x1x2}, 
\begin{align}
	\vec{x}_1-\vec{x}_2 &= 
	\vec{A}+ \kappa_1\vec{m_1}
	 -\vec{B}  - \kappa_2\vec{m_2} 
	 \\
	 &=
	\vec{A} 
	 -\vec{B}  + \vec{M} 
	\bm{\kappa}
\end{align}
which, upon substitution from 
        \eqref{eq:chapters/12/11/2/16/svd/M-svd}
	yields
\begin{align}
	\vec{x}_1-\vec{x}_2 &= 
	\vec{A} 
	 -\vec{B}  + 
\vec{U}\vec{\Sigma}\vec{V}^\top
\vec{V}\vec{\Sigma}^{-1}\vec{U}^\top\brak{\vec{B}-\vec{A}}
\\
	&=
	\brak{	\vec{A} 
	 -\vec{B}}  \brak{\vec{I}- 
\vec{U}\vec{\Sigma}
	\vec{\Sigma}^{-1}\vec{U}^\top}
\end{align}
			Thus, 
    \begin{align}
	    \norm{\vec{x}_1-\vec{x}_2} = 
	\norm{\brak{	\vec{A} 
	 -\vec{B}}  \brak{\vec{I}- 
\vec{U}\vec{\Sigma}
	    \vec{\Sigma}^{-1}\vec{U}^\top}}
        \label{eq:chapters/12/11/2/16/svd/min-sol}
    \end{align}
\item Least squares solution
	\begin{lstlisting}
	codes/book/skew_least.py
\end{lstlisting}
\item Least squares using builtin SVD 
	\begin{lstlisting}
	codes/book/skew_builtin.py
\end{lstlisting}
\item Code linking eigenvalues and singular values
	\begin{lstlisting}
	codes/book/skew_svd.py
\end{lstlisting}
\end{enumerate}

\end{enumerate}

