\begin{enumerate}
\item 
To check whether the given lines are skew,
from \eqref{eq:chapters/12/11/2/e11/params}
and 
	    \eqref{eq:chapters/12/11/2/16/lsq/rank},

\begin{align*}
\myvec{2&3&&1\\-1&-5&&0\\1&2&&-1}
\xleftrightarrow[R_3 \leftarrow R_3 - \frac{1}{2}R_1]{R_2 \leftarrow R_2 + \frac{1}{2}R_1}
\myvec{2&3&&1\\[1ex]0&-\frac{7}{2}&&\frac{1}{2}\\[1ex]0&\frac{1}{2}&&-\frac{3}{2}}\\
\xleftrightarrow{R_3 \leftarrow R_3 + 7R_2}
\myvec{2&3&&1\\[1ex]0&-\frac{7}{2}&&\frac{1}{2}\\[1ex]0&0&&-10}
\end{align*}
The rank of the matrix is 3. So the given lines are skew.
\item 
\begin{align}
\vec{M}^\top\vec{M} &= \myvec{2&-1&1\\3&-5&2}\myvec{2&3\\-1&-5\\1&2} \\ 
&= \myvec{6&13\\13&38} \label{eq:chapters/12/11/2/311/svd/MtM}
\end{align}
\begin{align}
\vec{MM}^\top &= \myvec{2&3\\-1&-5\\1&2}\myvec{2&-1&1\\3&-5&2}\\
&= \myvec{13&-17&8\\-17&26&-11\\8&-11&5} \label{eq:chapters/12/11/2/311/svd/MMt}
\end{align}
The characteristic polynomial of the matrix $\vec{MM}^\top$ is given by,
\begin{align}
\text{char}\brak{\vec{MM}^\top} &= \mydet{13-\lambda&-17&8\\-17&26-\lambda&-11\\8&-11&5-\lambda} \\
&= -\lambda^3 + 44\lambda^2-59\lambda
%\label{eq:chapters/12/11/2/311/svd/char-1}
\end{align}
resulting in 
\begin{align}
    \vec{U} &= \myvec{\frac{12-\sqrt{17}}{\sqrt{5}\sqrt{68-6\sqrt{17}}} & \frac{12+\sqrt{17}}{\sqrt{5}\sqrt{68+6\sqrt{17}}} & -\frac{3}{\sqrt{59}}\\
    \frac{1-3\sqrt{17}}{\sqrt{5}\sqrt{68-6\sqrt{17}}}&\frac{1+3\sqrt{17}}{\sqrt{5}\sqrt{68+6\sqrt{17}}} & \frac{1}{\sqrt{59}}\\
\frac{\sqrt{5}}{\sqrt{68-6\sqrt{17}}}&\frac{\sqrt{5}}{\sqrt{68+6\sqrt{17}}} & \frac{7}{\sqrt{59}} }
    \label{eq:chapters/12/11/2/311/svd/eig-params-1(a)}
\end{align}
and 
\begin{align}
	\vec{D}_1 &= \myvec{22+5\sqrt{17}&0&0\\0&22-5\sqrt{17}&0\\0&0&0}
    \label{eq:chapters/12/11/2/311/svd/eig-params-1(b)}
\end{align}
For $\vec{M}^\top\vec{M}$, the characteristic polynomial is
\begin{align}
    \text{char}\brak{\vec{M}^\top\vec{M}} &= \mydet{6-\lambda&13\\13&38-\lambda} \\&= \lambda^2-44\lambda+59
    \label{eq:chapters/12/11/2/311/svd/char-1}
\end{align}
Thus, the eigenvalues are given by
\begin{align}
    \lambda_1 = 22+5\sqrt{17},\ \lambda_2 = 22-5\sqrt{17}
\end{align}
resulting in 
\begin{align}
    \vec{V} &= \myvec{\frac{-16-5\sqrt{17}}{\sqrt{850+160\sqrt{17}}}&\frac{13}{\sqrt{850-160\sqrt{17}}}\\\frac{13}{\sqrt{850+160\sqrt{17}}}&\frac{-16+5\sqrt{17}}{\sqrt{850-160\sqrt{17}}}}
     \label{eq:chapters/12/11/2/311/svd/eig-params-2(a)}\\ 
	\vec{D}_2 &= \myvec{22-5\sqrt{17}&0\\0&22+5\sqrt{17}}
    \label{eq:chapters/12/11/2/311/svd/eig-params-2(b)}
\end{align}
Therefore, 
\begin{align}
    \vec{\Sigma} &= \myvec{\sqrt{22+5\sqrt{17}}&0\\0&\sqrt{22-5\sqrt{17}}\\0&0}
    \label{eq:chapters/12/11/2/311/svd/svd-params}
\end{align}
and substituting into 
        \eqref{eq:chapters/12/11/2/16/svd/min-sol},
\begin{align}
	\bm{\lambda} =  \myvec{\frac{25}{59}\\[1ex]-\frac{7}{59}}
\end{align}
which agrees with 
	\eqref{eq:chapters/12/11/2/e11/}.
\end{enumerate} 
