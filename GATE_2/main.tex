\let\negmedspace\undefined
\let\negthickspace\undefined
\documentclass[journal]{IEEEtran}
\usepackage[a5paper, margin=10mm, onecolumn]{geometry}
\usepackage{lmodern} % Ensure lmodern is loaded for pdflatex
\usepackage{tfrupee} % Include tfrupee package

\setlength{\headheight}{1cm} % Set the height of the header box
\setlength{\headsep}{0mm}     % Set the distance between the header box and the top of the text

\usepackage{gvv-book}
\usepackage{gvv}
\usepackage{cite}
\usepackage{amsmath,amssymb,amsfonts,amsthm}
\usepackage{algorithmic}
\usepackage{graphicx}
\usepackage{textcomp}
\usepackage{xcolor}
\usepackage{txfonts}
\usepackage{listings}
\usepackage{enumitem}
\usepackage{mathtools}
\usepackage{gensymb}
\usepackage{comment}
\usepackage[breaklinks=true]{hyperref}
\usepackage{tkz-euclide} 
\usepackage{listings}
\usepackage{gvv}                                       
\def\inputGnumericTable{}                                 
\usepackage[latin1]{inputenc}                                
\usepackage{color}                                            
\usepackage{array}                                            
\usepackage{longtable}                                       
\usepackage{calc}                                             
\usepackage{multirow}                                         
\usepackage{hhline}                                           
\usepackage{ifthen}                                           
\usepackage{lscape}
\usepackage{tikz}
\begin{document}

\bibliographystyle{IEEEtran}
\vspace{3cm}

\title{2008 CE 52-68}
\author{AI24BTECH11019-KOTHA PRATHEEK REDDY}

 \maketitle
%\newpage
% \bigskip
{\let\newpage\relax\maketitle}

\renewcommand{\thefigure}{\theenumi}
\renewcommand{\thetable}{\theenumi}
\setlength{\intextsep}{10pt} % Space between text and floats


\numberwithin{equation}{enumi}
\numberwithin{figure}{enumi}
\renewcommand{\thetable}{\theenumi}

\begin{enumerate}
%q1
	\item The wavefunction of which orbital is spherically symmetric: \hfill{[Gate 2017]}
\begin{multicols}{4}
	\begin{enumerate}
		\item $p_x$
		\item $p_y$
		\item $s$
		\item $d_{xy}$
	\end{enumerate}
\end{multicols}
%q2 
	\item The contour integral $\oint \frac{dz}{1+z^2}$ evaluated along a contour going from $- \infty$ to $\infty$ along the real axis and closed in the lower-half plane by a half circle is equal to $\underline{\hspace{2cm}}$.(up to two decimal places). \hfill{[Gate 2017]}
%q3
	\item The compton wavelength of a proton is $\underline{\hspace{2cm}}$ $fm$.(up to two decimal places).
	\brak{m_p = 1.67 \times 10^{-27}kg,h=6.626\times 10^{-34}Js,e=1.602\times 10^{-19}C,c=3\times 10^8 ms^{-1}} \hfill{[Gate 2017]}
%q4
\item Which one of the following conservation laws is violated in the decay $\tau^{+} \rightarrow \mu^+ \mu^+ \mu^-$ \hfill{[Gate 2017]}
\begin{multicols}{2}
	\begin{enumerate}
		\item Angular momentum
		\item Total Lepton number
		\item Electric charge
		\item Tau numeber
	\end{enumerate}
\end{multicols}
%q5
	\item Electromagnetic interactions are: \hfill{[Gate 2017]}
	\begin{multicols}{2}
	\begin{enumerate}
		\item $C$ non-conserving but $CP$ conserving
		\item $C$ conserving 
		\item $CP$ non-conserving but $CPT$ conserving
		\item $CPT$ non-conserving
	\end{enumerate}
\end{multicols}
%q6
	\item A one dimensional simple harmonic oscillator with Hamiltonian $H_0 = \frac{p^2}{2m} + \frac{1}{2}kx^2$ is unjected to a small perturbation. $H_1 = \alpha x + \beta x^2 + \gamma x^4$. The first order correction to the ground state energy is dependent on \hfill{[Gate 2017]}
	\begin{multicols}{4}
	\begin{enumerate}
		\item only $\beta$
		\item $\alpha$ and $\gamma$
		\item $\alpha$ and $\beta$
		\item only $\gamma$
	\end{enumerate}
\end{multicols}
%q7
	\item For the Hamiltonian $H = a_0 I + \overrightarrow b.\overrightarrow \sigma$ where $a_0 \in \mathbb{R}, \overrightarrow b$ is a real vector, $I$ is the $2 \times 2$ identity matrix, and $\overrightarrow \sigma$ are the Pauli matrices, the ground state energy is \hfill{[Gate 2017]}
	\begin{multicols}{4}
	\begin{enumerate}
		\item $|b|$
		\item $2a_0 - |b|$
		\item $a_0 - |b|$
		\item $a_0$
	\end{enumerate}
\end{multicols}
%q8
	\item The cofficient of $e^{ikx}$ in the Fourier expansion of $u\brak{x} = A \sin^{2}\brak{\alpha x}$ for $k = -2\alpha$ is \hfill{[Gate 2017]}
	\begin{multicols}{4}
	\begin{enumerate}
		\item $A/4$
		\item $-A/4$
		\item $A/2$
		\item $-A/2$
	\end{enumerate}
\end{multicols}
%q9
	\item The degeneracy of the third energy level of a 3-dimensional isotropic quantum harmonic oscillator is \hfill{[Gate 2017]}
	\begin{multicols}{4}
	\begin{enumerate}
		\item 6
		\item 12
		\item 8
		\item 10
	\end{enumerate}
\end{multicols}
%q10
	\item The electronic ground state energy of the Hydrogen aton is $-13.6eV$. The highest possible electronic energy eigenstate has an energy equal to \hfill{[Gate 2017]}
	\begin{multicols}{4}
	\begin{enumerate}
		\item 0
		\item $1eV$
		\item $+13.6eV$
		\item $\inf$
	\end{enumerate}
\end{multicols}
%q11
	\item A reversible Carnot engine is operated between temperatures $T_1$ and $T_2$ \brak{T_2 > T_1} with a photon gas as the working substance. The efficiency of the engine is \hfill{[Gate 2017]}
	\begin{multicols}{4}
	\begin{enumerate}
		\item $1-\frac{3T_1}{4T_2}$
		\item $1-\frac{T_1}{T_2}$
		\item $1-\brak{\frac{T_1}{T_2}}^{\frac{3}{4}}$
		\item $1-\brak{\frac{T_1}{T_2}}^{\frac{4}{3}}$
	\end{enumerate}
\end{multicols}
%q12
	\item In the nuclear reaction $^{13}C_6 + \nu _e \rightarrow ^{13}N_7 + X$, the particle $X$ is \hfill{[Gate 2017]}
\begin{multicols}{4}
	\begin{enumerate}
		\item an electron
		\item an anti-electron
		\item a muon
		\item a pion
	\end{enumerate}
\end{multicols}	

%q13
	\item Three charges \brak{2C,-1C,-1C} are placed at the vertices of an equilateral triangle of side $1m$ as shown in the figure. The component of the electric dipole moment about the marked origin along the $\hat y$ direction is $\underline{\hspace{2cm}}$ $Cm.$ \hfill{[Gate 2017]}
	\begin{tikzpicture}
    \draw (0,6) -- (0,0) -- (8.5,0);
    \node[align=center] at (-0.2,6) {y};
    \node[align=center] at (-0.1,-0.1) {0};
    \node[align = center] at (8.5,-0.2){x};
    \draw (4,0) -- (6,5);
    \draw (6,5) -- (8,0);
    \draw[dotted] (6,5) -- (6,0);
    \node[align=center] at (3.8,0.2) {-1$C$};
    \node[align=center] at (7.6,0.2) {-1$C$};
    \node[align=center] at (6,5.2) {2$C$};
     \draw[->, dotted, line width=0.5pt] (2.5,-0.4) -- (0,-0.4) ;   
    \draw[->, dotted, line width=0.5pt] (3.5,-0.4) -- (6,-0.4) ;
    \node[align = center] at (3,-0.4){$1.5m$};
     \draw[<-, dotted, line width=0.5pt] (6.371,5.148) -- (7.328,3.130) ;
     \draw[<-, dotted, line width=0.5pt] (8.5,0) -- (7.613,2.145) ;
     \node[align = center] at (7.4,2.7){$1.5m$};
    
\end{tikzpicture}

	
\end{enumerate}
\end{document}
