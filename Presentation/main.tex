\documentclass{beamer}
\mode<presentation>
\usepackage{amsmath}
\usepackage{amssymb}
%\usepackage{advdate}
\usepackage{adjustbox}
\usepackage{subcaption}
\usepackage{enumitem}
\usepackage{multicol}
\usepackage{mathtools}
\usepackage{listings}
\usepackage{minted}
\usepackage{xcolor}
\definecolor{bg}{rgb}{0.95,0.95,0.95}
\usepackage{url}
\def\UrlBreaks{\do\/\do-}
\usetheme{Boadilla}
\usecolortheme{lily}
\setbeamertemplate{footline}
{
  \leavevmode%
  \hbox{%
  \begin{beamercolorbox}[wd=\paperwidth,ht=2.25ex,dp=1ex,right]{author in head/foot}%
    \insertframenumber{} / \inserttotalframenumber\hspace*{2ex} 
  \end{beamercolorbox}}%
  \vskip0pt%
}
\setbeamertemplate{navigation symbols}{}

\providecommand{\nCr}[2]{\,^{#1}C_{#2}} % nCr
\providecommand{\nPr}[2]{\,^{#1}P_{#2}} % nPr
\providecommand{\mbf}{\mathbf}
\providecommand{\pr}[1]{\ensuremath{\Pr\left(#1\right)}}
\providecommand{\qfunc}[1]{\ensuremath{Q\left(#1\right)}}
\providecommand{\sbrak}[1]{\ensuremath{{}\left[#1\right]}}
\providecommand{\lsbrak}[1]{\ensuremath{{}\left[#1\right.}}
\providecommand{\rsbrak}[1]{\ensuremath{{}\left.#1\right]}}
\providecommand{\brak}[1]{\ensuremath{\left(#1\right)}}
\providecommand{\lbrak}[1]{\ensuremath{\left(#1\right.}}
\providecommand{\rbrak}[1]{\ensuremath{\left.#1\right)}}
\providecommand{\cbrak}[1]{\ensuremath{\left\{#1\right\}}}
\providecommand{\lcbrak}[1]{\ensuremath{\left\{#1\right.}}
\providecommand{\rcbrak}[1]{\ensuremath{\left.#1\right\}}}
\theoremstyle{remark}
\newtheorem{rem}{Remark}
\newcommand{\sgn}{\mathop{\mathrm{sgn}}}
\providecommand{\abs}[1]{\left\vert#1\right\vert}
\providecommand{\res}[1]{\Res\displaylimits_{#1}} 
\providecommand{\norm}[1]{\lVert#1\rVert}
\providecommand{\mtx}[1]{\mathbf{#1}}
\providecommand{\mean}[1]{E\left[ #1 \right]}
\providecommand{\fourier}{\overset{\mathcal{F}}{ \rightleftharpoons}}
%\providecommand{\hilbert}{\overset{\mathcal{H}}{ \rightleftharpoons}}
\providecommand{\system}{\overset{\mathcal{H}}{ \longleftrightarrow}}
	%\newcommand{\solution}[2]{\textbf{Solution:}{#1}}
%\newcommand{\solution}{\noindent \textbf{Solution: }}
\providecommand{\dec}[2]{\ensuremath{\overset{#1}{\underset{#2}{\gtrless}}}}
\newcommand{\myvec}[1]{\ensuremath{\begin{pmatrix}#1\end{pmatrix}}}
\let\vec\mathbf

\lstset{
%language=C,
frame=single, 
breaklines=true,
columns=fullflexible
}



\title{1.11.18}
\author{Kotha Pratheek Reddy\\AI24BTECH11019}
\date{\today} 
\begin{document}
   \begin{frame}
        \titlepage
    \end{frame}
    \begin{frame}
        \tableofcontents
    \end{frame}
    \section{Problem}
    \begin{frame}
        \frametitle{Problem Statement}
        Find the direction cosines of the line joining points $\vec{P}$  \myvec{4,3,-5} and $\vec{Q}$ \myvec{ -2, 1, 8}.
    \end{frame}
    \section{Solution}
    
        \begin{frame}
		\frametitle{Solution}
		  \begin{table}[H]
      \centering
      \begin{tabular}{|c|c|}
   \hline
   Point & Coordinate \\
   \hline
	$\vec{P}$ & $\myvec{4,3,-5}$ \\
   \hline
	$\vec{Q}$ & $\myvec{-2,1,8}$ \\
   \hline
\end{tabular}


      \caption{Coordinates}
	\label{}
     
  \end{table}

	\end{frame}
    \begin{frame}
	    \frametitle{Solution}
        Let the unit vector in the direction of the vector $\vec{PQ}$ be $\hat{a}$.Then \\
\begin{align}
    \hat{a} &= \frac{\vec{Q}-\vec{P}}{||\vec{Q}-\vec{P}||} \\
    \vec{P} &= \myvec{4 \\ 3 \\ -5} \\
    \vec{Q} &= \myvec{-2 \\ 1 \\8} \\
    \vec{Q}-\vec{P} &= \myvec{-6 \\ -2 \\ 13} \\
	||\vec{Q}-\vec{P}|| &= \sqrt{\brak{-6}^2 + \brak{-2}^2 + 13^2} \notag \\
    &= \sqrt{209} 
\end{align}
    \end{frame}
    \begin{frame}
	    \frametitle{Solution}
        From the above equations, \\
\begin{align}
    \hat{a} = \myvec{ \frac{-6}{\sqrt{209}} \\ \frac{-2}{\sqrt{209}} \\ \frac{13}{\sqrt{209}}}
\end{align}
The direction cosines of the the line joining $\vec{A}$ and $\vec{B}$ are the components of $\hat{a}$ i.e. $\frac{-6}{\sqrt{209}}$ , $\frac{-2}{\sqrt{209}} , \frac{13}{\sqrt{209}}$

    \end{frame}
\section{Plot}
\subsection{C-Code}
\begin{frame}[fragile]{C-Code}
    \begin{minted}[bgcolor=bg, linenos, fontsize=\small, breaklines]{c}
#include <math.h>
#include <stdlib.h>
#include <stdio.h>

typedef struct {
    double x;
    double y;
    double z;
} Vector;
\end{minted}
\end{frame}
 \begin{frame}[fragile]    \begin{minted}[bgcolor=bg, linenos, fontsize=\small, breaklines]{c}
	    // Function to calculate the direction cosines of the line joining two points
Vector* calculate_cosines(Vector* P, Vector* Q) {
    Vector* result = (Vector*)malloc(sizeof(Vector));

    // Calculate the direction vector
    double dx = Q->x - P->x;
    double dy = Q->y - P->y;
    double dz = Q->z - P->z;

    \end{minted}
\end{frame}
\begin{frame}[fragile]
	\begin{minted}[bgcolor=bg, linenos, fontsize=\small, breaklines]{c}
	   // Calculate the magnitude of the direction vector
    double magnitude = sqrt(dx * dx + dy * dy + dz * dz);
   

    
    // Calculate direction cosines
    result->x = dx / magnitude;  // cos(alpha)
    result->y = dy / magnitude;  // cos(beta)
    result->z = dz / magnitude;  // cos(gamma)

   

    return result;
}

// Function to free the allocated vector
void free_vector(Vector* vec) {
    free(vec);
}


    \end{minted}
\end{frame}

\subsection{Python Code}
\begin{frame}[fragile]{Python Code}
	\begin{minted}[bgcolor=bg, linenos, fontsize=\small, breaklines]{python}
		import numpy as np
import ctypes
import matplotlib.pyplot as plt

# Load the shared object file
lib = ctypes.CDLL('./code.so')

# Define the Point struct in Python
class Point(ctypes.Structure):
    _fields_ = [("x", ctypes.c_double),
                ("y", ctypes.c_double),
                ("z", ctypes.c_double)]

    \end{minted}
\end{frame}
\begin{frame}[fragile]
	\begin{minted}[bgcolor=bg, linenos, fontsize=\small, breaklines]{python}
       # Define the Vector struct in Python
class Vector(ctypes.Structure):
    _fields_ = [("x", ctypes.c_double),
                ("y", ctypes.c_double),
                ("z", ctypes.c_double)]
# Specify the return type and argument types for the calculate_cosines function
lib.calculate_cosines.restype = ctypes.POINTER(Vector)
lib.calculate_cosines.argtypes = [ctypes.POINTER(Point), ctypes.POINTER(Point)]


    \end{minted}
\end{frame}
\begin{frame}[fragile]
	 \begin{minted}[bgcolor=bg, linenos, fontsize=\small, breaklines]{python}
  # Function to draw angle between two vectors
def draw_angle_between_vectors(v1, v2, ax, text_offset=0):
    # Normalize the vectors
    v1 = v1 / np.linalg.norm(v1)
    v2 = v2 / np.linalg.norm(v2)

    # Compute the normal vector to the plane defined by v1 and v2
    normal = np.cross(v1, v2)
    normal = normal / np.linalg.norm(normal)  # Normalize the normal vector

    # Calculate the angle between the vectors
    angle_rad = np.arccos(np.dot(v1, v2))
    angle_deg = np.degrees(angle_rad)  # Convert to degrees

    # Parametrize the arc
    theta = np.linspace(0, angle_rad, 100)
    arc_points = np.array([np.cos(t) * v1 + np.sin(t) * np.cross(normal, v1) for t in theta]) / 2


    \end{minted}
\end{frame}
\begin{frame}[fragile]
	 \begin{minted}[bgcolor=bg, linenos, fontsize=\small, breaklines]{python}
     # Plot the arc
    ax.plot(arc_points[:, 0], arc_points[:, 1], arc_points[:, 2], 'g', label='Angle Arc')

    # Label the angle in the middle of the arc
    mid_arc_point = arc_points[len(arc_points) // 2] + (v1 + v2) / 5
    ax.text(mid_arc_point[0] + text_offset, mid_arc_point[1], mid_arc_point[2], f'{angle_deg:.0f}°', color='purple', fontsize=9)

# Create Point structs for P and Q
P = Point(4, 3, -5)
Q = Point(-2, 1, 8)


    \end{minted}
\end{frame}
\begin{frame}[fragile]
	 \begin{minted}[bgcolor=bg, linenos, fontsize=\small, breaklines]{python}
	# Call the C function to get direction cosines
vector_ptr = lib.calculate_cosines(ctypes.byref(P), ctypes.byref(Q))
origin = np.array([0, 0, 0])
vector = np.array([vector_ptr.contents.x, vector_ptr.contents.y, vector_ptr.contents.z]) * 2  # Scale for clarity

print("Direction cosines:")
print("Cos alpha:", vector_ptr.contents.x)
print("Cos beta:", vector_ptr.contents.y)
print("Cos gamma:", vector_ptr.contents.z)

# Plotting
fig = plt.figure()
ax = fig.add_subplot(111, projection='3d')
    \end{minted}
\end{frame}
\begin{frame}[fragile]
	 \begin{minted}[bgcolor=bg, linenos, fontsize=\small, breaklines]{python}
	# Plot the vector and axes
ax.quiver(*origin, *vector, length=1, color='r', label='Direction Vector')
ax.quiver(*origin, 0, 0, 2, length=1, color='k', label='Y-axis')
ax.quiver(*origin, 0, 2, 0, length=1, color='k', label='Z-axis')
ax.quiver(*origin, 2, 0, 0, length=1, color='k', label='X-axis')

# Draw angle arcs
draw_angle_between_vectors(np.array([1, 0, 0]), vector, ax)  # Angle with X-axis
draw_angle_between_vectors(np.array([0, 1, 0]), vector, ax)  # Angle with Y-axis
draw_angle_between_vectors(np.array([0, 0, 1]), vector, ax)  # Angle with Z-axis
	
    \end{minted}
\end{frame}
\begin{frame}[fragile]
 \begin{minted}[bgcolor=bg, linenos, fontsize=\small, breaklines]{python}
	# Set limits and labels
ax.set_xlim([-2, 2])
ax.set_ylim([-2, 2])
ax.set_zlim([-2, 2])
ax.set_xlabel('X-axis')
ax.set_ylabel('Y-axis')
ax.set_zlabel('Z-axis')

# Add axis labels
ax.text(1.2, 0, 0, "X", color='k')
ax.text(0, 1.2, 0, "Y", color='k')
ax.text(0, 0, 1.2, "Z", color='k')

plt.grid(True)
plt.legend()
plt.show()

# Free the C pointer
lib.free_vector(vector_ptr)
    \end{minted}
\end{frame}
	\begin{frame}
		\frametitle{Plot}
The codes in 
   \small
		\framebox{\url{https://github.com/Pratheek39/EE1030/tree/c703931a5fffd529b14ab319fcce2a0cf293937c/question2/Codes}}

plot the following figure

			
    \begin{figure}[h!]
   \centering
   \includegraphics[width=\linewidth]{Figures/Figure_1.png}
	  \caption{Line joining $\vec{P}$ and $\vec{Q}$}
   \label{stemplot}
\end{figure}
\end{frame}

   \end{document}
