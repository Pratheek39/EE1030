\documentclass[journal]{IEEEtran}
\usepackage[a5paper, margin=10mm]{geometry}
%\usepackage{lmodern} % Ensure lmodern is loaded for pdflatex
\usepackage{tfrupee} % Include tfrupee package


\setlength{\headheight}{1cm} % Set the height of the header box
\setlength{\headsep}{0mm}     % Set the distance between the header box and the top of the text


%\usepackage[a5paper, top=10mm, bottom=10mm, left=10mm, right=10mm]{geometry}

%
\usepackage{gvv-book}
\usepackage{gvv}
\setlength{\intextsep}{10pt} % Space between text and floats

\makeindex
\onecolumn
\begin{document}

\bibliographystyle{IEEEtran}
\vspace{3cm}

\title{Assignment-1}
\author{AI24BTECH11019-PRATHEEK}
\maketitle
\bigskip
\renewcommand{\thefigure}{\theenumi}
\renewcommand{\thetable}{\theenumi}

\section*{{C.Multipe Choice Questions}}
\begin{enumerate}
% q1
\item   Given positive integers $r>1,n>2$ and that {coefficient} of $\brak{3r}$ th terms in the binomial expansion of $\brak{1+x}^{2n}$ are equal. Then\hfill{${(1983-1Mark)}$}
\begin{enumerate}\begin{multicols}{2}
\item $n=2r$
\item $n=2r+1$
\item $n=3r$
\item none of these
\end{multicols}
\end{enumerate}
% q2
\item The coefficient of $x^4$ in $\brak{\frac{x}{2}-\frac{3}{x^2}}^{10}$ is \hfill{{\brak{1983-1 Mark) }}
\begin{enumerate}
\begin{multicols}{2}
\item $\frac{405}{256}$
\item $\frac{504}{259}$
    
\item $\frac{450}{263}$
\item none of these
\end{multicols}
\end{enumerate}
% q3
\item The expression $\brak{x+\brak{x^3-1}^{\frac{1}{2}}}^5$ + $\brak{x-\brak{x^3-1}^{\frac{1}{2}}}^5$ is a polynomial of degree\hfill{$\brak{1992-2Marks}$}
\begin{enumerate}
\begin{multicols}{2}
    

    \item 5
    \item 6
    \item 7
    \item 8
    \end{multicols}
\end{enumerate}
%q4
\item If in the expansion of $\brak{1+x}^m\brak{1-x})^n$,the coefficients of $x$ and $x^2$ are $3$ and $-6$ respectively, then $m$ is
{\hfill$\brak{1999-2Marks}$}
\begin{enumerate}
\begin{multicols}{2}
    \item 6
    \item 9
    \item 12
    \item 24
    \end{multicols}
\end{enumerate}
%q5
\item For $2\leq r\leq n$, $\comb{n}{r}$ + $2\comb{n}{r-1}$ + $\comb{n}{r-2}$ =\hfill{$\brak{2000S}$}
\begin{enumerate}\begin{multicols}{2} 
    \item $\comb{n+1}{r-1}$ \item $2\comb{n+1}{r+1}$
    \item $2\comb{n+2}{r}$  \item $\comb{n+2}{r}$
    \end{multicols}
\end{enumerate}
%q6
\item In the binomial expansion of $\brak{a-b}^n$,$n\geq 5,t$ the sum of of the $5^{th}$ and $6^{th}$ terms is zero.Then $a/b$  equals
{\hfill $\brak{2001S}$}
\begin{enumerate}
\begin{multicols}{2}
    \item $\brak{n-5}/6$
    \item $\brak{n-4}/5$
    \item $5/\brak{n-4}$
    \item $6/\brak{n-5}$
 \end{multicols}   
\end{enumerate}
%q7
\item The sum $\sum_{i=0}^{9} \comb{10}{i}\comb{20}{m-i}$,(where$\comb{p}{q}=0$ if {
$p<q$)}is maximum when $m$ is{\hfill$\brak{2002S}$}
\begin{enumerate}
\begin{multicols}{2}
    \item5 
    \item10
    \item15
    \item 20
    \end{multicols}
\end{enumerate}
%q8
\item Coefficient of $t^{24}$ in $\brak{{1+t^2}}^{12}\brak{1+t^{12}}\brak{1+t^{24}}$ is{\hfill$\brak{2003S}$}
\begin{enumerate}
\begin{multicols}{2}
\item $\comb{12}{6}$+3
\item $\comb{12}{6}$+1 
\item $\comb{12}{6}$
\item $\comb{12}{6}$+2\end{multicols}
\end{enumerate}
%q9
\item If 
\begin{align}
\comb{n-1}{r} &= \brak{k^2-3} \comb{n}{r+1} \notag
\end{align}
then ($k \in $ )\hfill{$\brak{2004S}$}

\begin{enumerate}
\begin{multicols}{2}
    \item $(-8,-2]$
    \item $[2,\infty)$
    \item$\sbrak{-\sqrt{3},\sqrt{3}}$
    \item$(\sqrt{3},2]$
    \end{multicols}
\end{enumerate}

\item The value of
	$\comb{30}{0}$$\comb{30}{10}$-$\comb{30}{1}$$\comb{30}{11}$+$\comb{30}{2}$$\comb{30}{12}$\dots$\comb{30}{20}$$\comb{30}{30}$ is where $\comb{n}{r}$ = $\comb{n}{r}${\hfill$\brak{2005S}$}
\begin{enumerate}
\begin{multicols}{2}
\item$\comb{30}{10}$ 
\item$\comb{30}{15}$
\item$\comb{60}{30}$ 
\item$\comb{31}{10}$
\end{multicols}
\end{enumerate}
\item For $r=0,1\cdots,10$, let $A_r,B_r$ and $C_r$ denote,respectively the coefficients of $x^r$ in the expansions of $\brak{1+x}^{10}$,$\brak{1+x}^{20}$and$\brak{1+x}^{30}$.Then $\sum_{r=1}^{10}A_r\brak{B_{10}B_r-C{10}A_r}$ is equal to{\hfill$\brak{2010}$}
\begin{enumerate}
 \begin{multicols}{2}
 \item $B_{10}-C_{10}$ 
 \item $A_{10}\brak{B_{10}^2C_{10}A_{10}}$ 
 \item $0$
 \item $C_{10}-B_{10}$
 \end{multicols}
\end{enumerate}

\item  Coefficient of $x^{11}$ in the expansion of$ \brak{1+x^2}^4 \brak{1+x^3}^7 \brak{1+x^4}^{12}$ is \hfill \brak {JEE Adv. 2014}

\begin{enumerate}
\begin{multicols}{2}
 \item 1051
 \item 1106
 \item 1113
 \item 1120
 \end{multicols}
\end{enumerate}
\end{enumerate}
\section*{D.MCQs with One or More than One Correct}
\begin{enumerate}
	\item If $c_r$ stands for $\comb{n}{r}$ ,the the sum of the series $\frac{2\brak{\frac{n}{2}!}\brak{\frac{n}{2}!}}{n!}\sbrak{C_0^2-2C_1^2+3C_2^2-\dots+\brak{-1}^n\brak{n+1}C_n^2}$, where n is an even positive integer is equal to{{$\hfill\brak{1992-2Marks}$}}
\begin{enumerate}
\begin{multicols}{2}
\item 0
\item $\brak{-1}^{\frac{n}{2}}\brak{n+1}$
\item $\brak{-1}^{\frac{n}{2}}\brak{n+2}$
\item $\brak{-1}^n n$
\item none of these
\end{multicols}
\end{enumerate}
\item If $a_n = \sum_{r=0}^{n}\frac{1}{\comb{n}{r}}$,then $\sum_{r=0}^{n}\frac{r}{\comb{n}{r}}$ equals {{$\hfill\brak{1998-2Marks}$}}
\begin{enumerate}
\begin{multicols}{2}
\item $\brak{n-1}a_n$
\item $na_n$
\item $\frac{1}{2}na_n$
\item None of The above
\end{multicols}
\end{enumerate}
\end{enumerate}
\end{document}

